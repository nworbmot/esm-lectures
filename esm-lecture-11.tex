% NB: use pdflatex to compile NOT pdftex.  Also make sure youngtab is
% there...

% converting eps graphics to pdf with ps2pdf generates way too much
% whitespace in the resulting pdf, so crop with pdfcrop
% cf. http://www.cora.nwra.com/~stockwel/rgspages/pdftips/pdftips.shtml




\documentclass[10pt,aspectratio=169,dvipsnames]{beamer}

\usetheme[color/block=transparent]{metropolis}

\usepackage[absolute,overlay]{textpos}
\usepackage{booktabs}
\usepackage[utf8]{inputenc}

\usepackage{tikz}


\usepackage[scale=2]{ccicons}

\usepackage[official]{eurosym}


%use this to add space between rows
\newcommand{\ra}[1]{\renewcommand{\arraystretch}{#1}}


\newcommand{\R}{\mathbb{R}}


\setbeamerfont{alerted text}{series=\bfseries}
\setbeamercolor{alerted text}{fg=Mahogany}
\setbeamercolor{background canvas}{bg=white}



\def\l{\lambda}
\def\m{\mu}
\def\d{\partial}
\def\cL{\mathcal{L}}
\def\co2{CO${}_2$}
\def\bra#1{\left\langle #1\right|}
\def\ket#1{\left| #1\right\rangle}
\newcommand{\braket}[2]{\langle #1 | #2 \rangle}
\newcommand{\norm}[1]{\left\| #1 \right\|}
\def\corr#1{\Big\langle #1 \Big\rangle}
\def\corrs#1{\langle #1 \rangle}


\def\mw{\text{ MW}}
\def\mwh{\text{ MWh}}
\def\emwh{\text{ \euro/MWh}}

\newcommand{\ubar}[1]{\text{\b{$#1$}}}


% for sources http://tex.stackexchange.com/questions/48473/best-way-to-give-sources-of-images-used-in-a-beamer-presentation

\setbeamercolor{framesource}{fg=gray}
\setbeamerfont{framesource}{size=\tiny}


\newcommand{\source}[1]{\begin{textblock*}{5cm}(10.5cm,8.35cm)
    \begin{beamercolorbox}[ht=0.5cm,right]{framesource}
        \usebeamerfont{framesource}\usebeamercolor[fg]{framesource} Source: {#1}
    \end{beamercolorbox}
\end{textblock*}}

\usepackage{hyperref}

\usepackage{tikz}


\usepackage[europeanresistors,americaninductors]{circuitikz}


%\usepackage[pdftex]{graphicx}


\graphicspath{{graphics/}}

\DeclareGraphicsExtensions{.pdf,.jpeg,.png,.jpg,.gif}



\def\goat#1{{\scriptsize\color{green}{[#1]}}}



\let\olditem\item
\renewcommand{\item}{%
\olditem\vspace{5pt}}

\title{Energy System Modelling\\ Summer Semester 2020, Lecture 11}
%\subtitle{---}
\author{
  {\bf Dr. Tom Brown}, \href{mailto:tom.brown@kit.edu}{tom.brown@kit.edu}, \url{https://nworbmot.org/}\\
  \emph{Karlsruhe Institute of Technology (KIT), Institute for Automation and Applied Informatics (IAI)}
}

\date{}


\titlegraphic{
  \vspace{0cm}
  \hspace{10cm}
    \includegraphics[trim=0 0cm 0 0cm,height=1.8cm,clip=true]{kit.png}

\vspace{5.1cm}

  {\footnotesize

  Unless otherwise stated, graphics and text are Copyright \copyright Tom Brown, 2020.
  Graphics and text for which no other attribution are given are licensed under a
  \href{https://creativecommons.org/licenses/by/4.0/}{Creative Commons
  Attribution 4.0 International Licence}. \ccby}
}

\begin{document}

\maketitle


\begin{frame}

  \frametitle{Table of Contents}
  \setbeamertemplate{section in toc}[sections numbered]
  \tableofcontents[hideallsubsections]
\end{frame}


\section{Present value and discounting}


\begin{frame}
  \frametitle{The value of money depends on time}


  \alert{Question 1:} What would you prefer: \euro 1000 today, or \euro 1000 in 3 years?

  \pause

  \euro 1000 today can be invested in the bank with an interest rate of 5\%.

  After 3 years you would have
  \begin{equation*}
     1000 \cdot (1 + 0.05)^3 = 1158
  \end{equation*}

  \alert{Answer 1:} Best to take the money today and use the opportunity to invest!

  \vspace{.5cm}

 ``Money in the future is worth less than money today.''

\end{frame}


\begin{frame}
  \frametitle{The value of money depends on time}


  \alert{Question 2:} What would you prefer: \euro 1000 today, or \euro 1300 in 5  years?

  \pause

  If you invested \euro 1000 today, after 5 years you would have only
  \begin{equation*}
     1000 \cdot (1 + 0.05)^5 = 1276
  \end{equation*}

  \alert{Answer 2:} Best to wait for the \euro 1300 in 5 years!


\end{frame}



\begin{frame}
  \frametitle{Present value}

  To allow comparison between income and outgoings in different years, we need to agree on a particular point in time to evaluate the cash flows.

  The simplest and most frequently used time point: today's value, known as the \alert{present value}.

  For an \alert{interest rate} $r$ we multiply the income or outgoings in year $t$ by the \alert{discount factor}
  \begin{equation*}
    \frac{1}{(1+r)^t}
  \end{equation*}
  to calculate the present value. We have \alert{discounted} the future cash flow.

  Future income or outgoings are \alert{worth less} from today's point of view (as long as $r$ is positive).

  ``Money in the future is worth less than money today.''

\end{frame}

\begin{frame}
  \frametitle{Example: present value}

  For our example with interest rate 5\% we can now order the options:
  \ra{1.1}
  \begin{table}[!t]
    \begin{tabular}{rrr}
      \toprule
      Income (\euro) & Year & Present value (\euro)  \\
      \midrule
      1000  &  3 & $\frac{1000}{(1+0.05)^3} = 863$  \\
      1000  &  0 & $\frac{1000}{(1+0.05)^0} = 1000$ \\
      1300  &  5 & $\frac{1300}{(1+0.05)^5} = 1019$ \\
      \bottomrule
    \end{tabular}
  \end{table}

\end{frame}



\section{Investment calculations}


\begin{frame}
  \frametitle{Motivation: Power plant investment}

  A company is considering investing in a photovoltaic plant on its roof. The key figures:


  \begin{columns}[T]
\begin{column}{6cm}

    \begin{table}[!t]
    \begin{tabular}{lr}
      \toprule
      Size &  100 kW \\
      Investment cost & 800 \euro kW$^{-1}$ \\
      Operating cost & 20 \euro kW$^{-1}$ a$^{-1}$\\
      Feed-In Tariff & 0.1 \euro kWh$^{-1}$\\
      Full load hours & 1000 \\
      Period of subsidy & 20 years \\
      \bottomrule
    \end{tabular}
  \end{table}
\end{column}
\begin{column}{6cm}

    \includegraphics[width=6.5cm]{PV_Anlage_auf_der_Rottweiler_Stadthalle.jpg}
\end{column}
  \end{columns}

  \vspace{.5cm}

  The company can invest its money elsewhere for a return of 5\%.

  \alert{Is it worthwhile to invest in the photovoltaic plant?}

  \source{\href{https://www.ise.fraunhofer.de/content/dam/ise/de/documents/publications/studies/DE2018_ISE_Studie_Stromgestehungskosten_Erneuerbare_Energien.pdf}{Fraunhofer ISE Stromgestehungskosten 2018}, \href{https://de.wikipedia.org/wiki/ENRW_Energieversorgung_Rottweil\#/media/Datei:PV_Anlage_auf_der_Rottweiler_Stadthalle.jpg}{Wikipedia}}

\end{frame}

\begin{frame}
  \frametitle{Investment calculations}

  An \alert{investment calculation} quantifies the financial costs and benefits of an investment, assuming that future income and outgoings can be predicted.

  It considers
  \begin{itemize}
  \item \alert{Capital costs} - Costs for investments and installation
  \item \alert{Consumption costs} - Fuel, other materials (e.g. lubricants for wind turbine), etc.
  \item \alert{Operating costs} - Maintenance, wages, insurance, management, etc.
  \item \alert{Income} - depends on market price, subsidies, and production
  \end{itemize}

\end{frame}



\begin{frame}
  \frametitle{Dynamic investment calculation}

  For a \alert{dynamic investment calculation} we sum the present values of all income and outgoings over the $T$ years of operation taking account of the interest rate $r$ to get the \alert{Net Present Value (NPV)}:
  \begin{equation*}
    NPV = \sum_{t=0}^T \frac{ - I_t - V_t - B_t + U_t}{(1+r)^t}
  \end{equation*}
  where $I_t$ is the capital expenditure in year $t$, $V_t$ the
  consumption costs (e.g. for fuel cost $o_t$ and annual production $Q_t$, $V_t = o_t\cdot Q_t$), $B_t$ the operating costs und $U_t$ the income (e.g. average market value $\lambda_t$ times annual production
  $Q_t$, $U_t = \lambda_t\cdot Q_t$).

  \alert{Conclusion:} If $NPV > 0$, the investment is worthwhile.

  If $NPV < 0$, better to invest with a rate of return of $r$ elsewhere.

  For comparisons between different investments, a higher NPV should be preferred.
\end{frame}


\begin{frame}
  \frametitle{Example: Rooftop photovoltaic unit}

    All cash flows (costs and income) in \euro:
    \begin{table}[!t]
    \begin{tabular}{lrrrrr}
      \toprule
      year $t$ &  0 & 1 & 2 & $\cdots$ & 20 \\
      \midrule
      Capital costs $I_t$ & 80,000  & 0 & 0  & & 0\\
      Operating costs $B_t$ & 0 & 2,000 & 2,000 & & 2,000  \\
      Income $U_t$ & 0 & 10,000  & 10,000 & & 10,000 \\
      \midrule
      Net cash flow $U_t - I_t - B_t$ & -80,000 & 8,000 & 8,000 & & 8,000 \\
      \midrule
      Discount factor $\frac{1}{(1+r)^t}$ & 1 & $\frac{1}{(1+r)}$ & $\frac{1}{(1+r)^2}$ & &  $\frac{1}{(1+r)^{20}}$ \\
      \bottomrule
    \end{tabular}
    \end{table}


\end{frame}



\begin{frame}
  \frametitle{NPV simplification}

  If investments only occur in the first year, and the costs and income for the following years are constant, we can simplify the NPV formula:
  \begin{equation*}
    NPV = -I_0 + (U - V - B) \sum_{t=1}^T \frac{1}{(1+r)^t}
  \end{equation*}
  The sum $\sum$ is called the \alert{Present Value Factor} $PVF(r,T)$.

  For a geometric series with $|q| < 1$ we have $\sum_{n=0}^\infty q^n = \frac{1}{1-q}$.
  For $q= (1+r)^{-1}$ we can simplify the formula
  \begin{align*}
    PVF(r,T) & =  \sum_{t=1}^T \frac{1}{(1+r)^t} \\
             & = \left[\frac{1}{(1+r)} - \frac{1}{(1+r)^{T+1}}  \right] \sum_{t=0}^\infty \frac{1}{(1+r)^t}
              = \left[\frac{1}{(1+r)} - \frac{1}{(1+r)^{T+1}}  \right] \frac{1}{1 - (1+r)^{-1}} \\
             & = \left[\frac{1}{(1+r)} - \frac{1}{(1+r)^{T+1}}  \right] \frac{1+r}{1 + r  - 1}  = \frac{1}{r} \left[1  - \frac{1}{(1+r)^{T}}  \right]
  \end{align*}

\end{frame}


\begin{frame}
  \frametitle{Example: Rooftop photovoltaic unit}

  For our example with $r= 0.05$
  \begin{align*}
    NPV & = -80,000 + (10,000 - 2,000)  \cdot \frac{1}{r} \left[1  - \frac{1}{(1+r)^{T}}  \right] \\
    & = -80,000 + 8,000*12.5 \\
    & = 19,698
  \end{align*}

  \alert{Conclusion:} It's worthwhile to invest in the photovoltaic unit!

  \pause

  NB: The calculation is very sensitive to the interest rate, e.g. with $r = 0.08$
  \begin{align*}
    NPV & = -80,000 + 8,000*9.8 \\
    & = -1,454
  \end{align*}

  \alert{Conclusion:} The investment is not worthwhile.

\end{frame}

\begin{frame}
  \frametitle{Return On Investment (ROI)}

  The expected return or \alert{Return On Investment (ROI)} is the required interest rate to reach the point $NPV = 0$.

  In our example you can either experiment or use the Newton-Raphson algorithm to determine the ROI $r$
  \begin{equation*}
    0 = NPV = -I_0 + (U - V - B) \sum_{t=1}^T \frac{1}{(1+r)^t}
  \end{equation*}

  In our example we find an ROI of $r=7.75$\%.

\end{frame}



\begin{frame}
  \frametitle{German example figures for electricity production technologies in 2018}

  WACC is the \alert{Weighted Average Cost of Capital} over the bank interest rate for borrowed capital (Fremdkapital) and the investor's ROI on their own investment (Eigenkapital).

  \centering
  \includegraphics[width=14cm]{stromgestehungskosten.png}


  \source{\href{https://www.ise.fraunhofer.de/content/dam/ise/de/documents/publications/studies/DE2018_ISE_Studie_Stromgestehungskosten_Erneuerbare_Energien.pdf}{Fraunhofer ISE Stromgestehungskosten 2018}}

\end{frame}


\begin{frame}
  \frametitle{Warning: Discounting over long time periods}

  Over long time periods the discounting can have a very large effect....

  \begin{columns}[T]
\begin{column}{9cm}
  \centering
  \includegraphics[width=9.5cm]{discounting.pdf}
\end{column}
\begin{column}{5cm}
  \begin{itemize}
  \item Long-term benefits aren't seen, e.g. long production life of nuclear power plants or benefits of long-lived efficiency measures
  \item Long-term costs are also suppressed, e.g. decommissioning, waste disposal, climate damages
  \item This is a \alert{controversial topic!}
  \end{itemize}
\end{column}
  \end{columns}

\end{frame}



\begin{frame}
  \frametitle{Programming example: photovoltaic plant}

  \centering
  \includegraphics[width=10.5cm]{PV_programming_example.png}


\end{frame}

\begin{frame}
  \frametitle{Programming example: nuclear plant}

  \centering
  \includegraphics[width=10.5cm]{nuclear_programming_example.png}


\end{frame}




\begin{frame}
  \frametitle{Summary}

  \begin{itemize}
  \item Future income or costs are worth less from today's point of view
  \item To calculate the \alert{present value} give the \alert{interest rate} $r$, multiply the cash flow in year $t$ by the \alert{discount factor}  $\frac{1}{(1+r)^t}$
  \item To calculate the \alert{net present value (NPV)} for an investment, sum the present values of all income and costs
  \item If  $NPV > 0$, the investment is worthwhile compared to investing with interest rate $r$
  \item For two different investments, a higher NPV should be preferred
  \item \alert{Long-term} costs or benefits are \alert{suppressed} by discounting
  \end{itemize}

\end{frame}


\section{Levelised Cost Of Electricity (LCOE)}




\begin{frame}
  \frametitle{Levelised Cost Of Energy (LCOE)}

  You can also solve for the market value or feed-in tariff that's
  necessary to cover all the costs of the investment, i.e. the point
  where the present value of all income balances the present value of
  all costs.  You solve for the price $\lambda$ such that
  \begin{equation*}
   0 = NPV = -I_0 + (\lambda Q - oQ - B) PVF(r,T)
  \end{equation*}
  (using $V = oQ$). We find:
  \begin{equation*}
    \lambda = \frac{1}{Q} \left( \frac{I_0}{PVF(r,T)} + B + oQ \right) = \frac{1}{Q} \left( \frac{I_0}{PVF(r,T)} + B  \right) + o
  \end{equation*}

  In our example we find a price of $\lambda =$ 89 \euro/MWh for $i=0.05$. % One could reduce the feed-in tariff from 100 \euro/MWh to this value!

  This value corresponds to the average long-term costs of the unit, since we've divided the total yearly costs by the total production $Q$. It is called the
the \alert{Levelised Cost Of Energy (LCOE)}.

  It is also called the  \alert{Long-Run Marginal Cost (LMRC)}, since we've added to the short-run marginal cost $o$ an annualised contribution to the capital cost and the operating costs.

  Check: The higher $I_0$ or $B$ are, the higher the LCOE. The higher $Q$ is, the lower the LCOE.
\end{frame}



\begin{frame}
  \frametitle{Annuity}

  The \alert{annuity} is the annualised investment cost $a = \frac{I_0}{PVF(r,T)}$ and
  $a(r,T) = \frac{1}{PVF(r,T)}$ is the \alert{annuity factor}, which spreads the capital costs $I_0$ evenly over the operational years of the investment (like a mortgage for a house).

  For a loan $I_0$ from the bank, the bank is compensated for the \alert{opportunity cost} of investing elsewhere at a rate of $r$ by an annual fixed sum $a$ so that the NPV for the bank is zero
  \begin{equation*}
    0 = NPV  = -I_0 + \sum_{t=0}^T \frac{a}{(1+r)^t} = -I_0 + PVF(r,T) \frac{I_0}{PVF(r,T)}
  \end{equation*}

  The formula for the annuity factor is derived from that for the PVF:
  \begin{equation*}
     a(r,T)  = \frac{1}{PVF(r,T)} = \frac{ r}{ 1 - (1+r)^{-T}}
  \end{equation*}


\end{frame}



\begin{frame}[fragile]
  \frametitle{Examples of annuity factor}

  \begin{columns}[T]
\begin{column}{6.5cm}

  AF = Annuity Factor, $a(r,T)$
  \begin{table}[!t]
	\centering
	\begin{tabular}{@{}rrr@{}}
\toprule
Lifetime $T$  &  Discount Rate $r$ & AF $a(r,T)$ \\
years & \% & per unit \\
\midrule
20 & 0 & 0.05 \\
20 & 5 & 0.08 \\
20 & 10 & 0.12 \\
20 & 20 & 0.21 \\
40 & 0 & 0.025 \\
40 & 5 & 0.06 \\
40 & 10 & 0.10 \\
40 & 20 & 0.20 \\
\bottomrule
	\end{tabular}
\end{table}
\end{column}
\begin{column}{7.5cm}
  Things to notice:
  \begin{itemize}
  \item AF reduce to $1/T$ in limit $r \to 0$
  \item AF climbs steeply with $r$
  \item For long lifetimes, AF is similar to short lifetimes for high $r$ - in reality investors try to pay off investments faster than lifetime
  \item In reality, an investor would provide some capital themselves, e.g. 10-20\% of the capital cost, and borrow the rest from the bank. The weighted average of the investor's desired internal rate of return and that of the bank loan is the \alert{weighted average cost of capital} (WACC).
  \end{itemize}
\end{column}
  \end{columns}
\end{frame}




\begin{frame}[fragile]
  \frametitle{Parameters for different generation technologies}

  Here are some typical investment and operational parameters projected for 2020:
  \vspace{-.5cm}
  \begin{table}[!t]
	\centering
	\begin{tabular}{@{}lrrrrrrr@{}}
\toprule
Source   & Lifetime & Capital Cost &  Fix O\&{}M &  Var O\&{}M & $\eta$ & Fuel Cost & Marg. Cost \\
& years & \euro kW$^{-1}$  & \euro kW$^{-1}$a$^{-1}$ &  \euro MWh$_{\textrm{el}}^{-1}$ & [\%] & \euro/MWh$_{\textrm{th}}$ & \euro/MWh$_{\textrm{el}}$ \\
\midrule
Hard Coal    & 40 & 1200 & 30 & 6 & 39 & 10 & 32 \\
Gas OCGT     & 30 & 400  & 15 & 3 & 39 & 20 & 54 \\
Gas CCGT     & 30 & 800  & 20 & 4 & 60 & 20 & 37 \\
Nuclear      & 40-60 & 6000  & 0 & 6 & 33 & 3.3 & 16 \\
Wind Onshore & 25 & 1240 & 35 & 0 & & 0 & 0 \\
Solar PV     & 25 & 750  & 25 & 0 & & 0 & 0 \\
\bottomrule
	\end{tabular}
\end{table}
O\&{}M = Operation and Maintenance, Var. = Variable, Fix. = Fixed, $\eta$ = efficiency

For a plant with capacity $G_s$ in MW and yearly production $Q$ in MWh$_{\textrm{el}}$, we have $I_0 = 1000 \cdot G_s \cdot (\textrm{Capital Cost})$, $B = 1000\cdot G_s \cdot (\textrm{Fix O\&{}M})$, $V = Q\cdot o$ where o is the marginal cost $o =  (\textrm{Marg. Cost}) = (\textrm{Var O\&{}M}) + (\textrm{Fuel Cost})/\eta$.

\source{\href{https://www.diw.de/documents/publikationen/73/diw_01.c.424566.de/diw_datadoc_2013-068.pdf}{DIW Data Documentation, 2013}}
\end{frame}




\begin{frame}
  \frametitle{LCOE for dispatchable generators depends on capacity factor}

  The LCOE had the form (Marg. Cost) + (Yearly Fixed Costs)/(Yearly Production). Therefore it decreases with increasing capacity factor:

  \begin{columns}[T]
    \begin{column}{8.5cm}
      \vspace{.5cm}
  \centering
  \includegraphics[width=8cm]{lcoe_cf.pdf}
\end{column}
\begin{column}{6.5cm}
  \begin{itemize}
  \item LCOE $>$ marginal cost
  \item LCOE starts high then reduces as fixed costs are spread over more hours
  \item There are \alert{crossing points} where some types of generators become cheaper for a given capacity factor
  \item NB: All generators need downtime for regular maintenance, so cf $< 0.9$
  \item NB: Carbon pricing would alter this graphic by adding to the marginal cost
  \end{itemize}
\end{column}
  \end{columns}

\end{frame}


\begin{frame}{LCOE for wind and solar depends on location: worldwide auction results 2017}

  \centering
  \includegraphics[width=11.5cm]{baringa-auctions.png}

  \source{\href{https://www.baringa.com/getmedia/99d7aa0f-5333-47ef-b7a8-1ca3b3c10644/Baringa_Scottish-Renewables_UK-Pot-1-CfD-scenario_April-2017_Report_FINA/}{Baringa Partners LLP 2017}}

\end{frame}


\begin{frame}[fragile]
  \frametitle{Levelised Cost of Electricity Since 2009 in US}

  NB: Treat with care since LCOE doesn't take account of time or place of generation!

  \centering
  \includegraphics[width=14cm]{lazard-cropped.jpg}

  \source{\href{https://www.lazard.com/perspective/levelized-cost-of-energy-2017/}{Lazard's LCOE Analysis V11}}

\end{frame}



\section{Multi-horizon investment: Motivation}


\begin{frame}
  \frametitle{Short-run efficiency}

  Short-run efficiency is concerned with the \alert{efficient operation} of the existing energy system, assuming that the capacities of all investments are fixed.
  \begin{columns}[T]
    \begin{column}{7.5cm}

      \alert{Example}: Power plant dispatch for inelastic demand $d$. All capacities $G_s$ [MW] are fixed. We optimise the \alert{dispatch} $g_s$ [MW], assuming that the \alert{marginal costs} $o_s$ [\euro/MWh] scale linearly with the dispatch. We minimise \alert{total operational costs}:
      \begin{equation*}
        \min_{\{g_s\}} \sum_s o_s g_s
      \end{equation*}
      with constraints
  \begin{align*}
    \sum_s g_s & = d\hspace{1cm} \leftrightarrow \hspace{1cm} \l \\
        g_s  & \leq  G_s  \hspace{1cm}\leftrightarrow\hspace{1cm} \bar{\m}_s \\
    - g_s  & \leq  0  \hspace{1cm}\leftrightarrow\hspace{1cm} \ubar{\m}_s
  \end{align*}
    \end{column}
    \begin{column}{6.5cm}

        \vspace{.5cm}
        %https://www.forum-csr.net/News/7186/Marktwirtschaftparadox.html
        \includegraphics[trim=0 0cm 0 0cm,width=7cm,clip=true]{merit-order-effekt_neu1.png}


    \end{column}
  \end{columns}

\end{frame}

\begin{frame}
  \frametitle{Long-run efficiency}


  Long-run efficiency is concerned with the \alert{efficient operation} and \alert{the efficient dimensioning of investments} in the energy system.

  \begin{columns}[T]
    \begin{column}{7.5cm}

      \alert{Example}: Power plant \alert{dispatch} $g_{s,t}$ (costs $o_s$) and \alert{capacities} $G_s$ (annualised costs $c_s$) are optimised over a year of hourly time periods $t$ with demand $d_t$:
      \begin{equation*}
        \min_{\{g_{s,t},G_s\}} \sum_{s,t} o_s g_{s,t} + \sum_s c_s G_s
      \end{equation*}
      with constraints
  \begin{align*}
    \sum_s g_{s,t} & = d_t\hspace{1cm} \leftrightarrow \hspace{1cm} \l_t \\
        g_{s,t}   & \leq  G_s  \hspace{1cm}\leftrightarrow\hspace{1cm} \bar{\m}_{s,t} \\
    - g_{s,t}  & \leq  0  \hspace{1cm}\leftrightarrow\hspace{1cm} \ubar{\m}_{s,t}
  \end{align*}
    \end{column}
    \begin{column}{7cm}

  \begin{tikzpicture}
\node[anchor=south west,inner sep=0] (image) at (0,0) {\includegraphics[width=6cm]{screening-load-duration-clean}};
\draw (1.8,4.0) node{$c_2$};
\draw (1.8,5.2) node{$c_1$};
\draw (2.5,3.3) node{$\theta_2$};
\draw (4.8,3.3) node{$\theta_1$};
\draw (2.5,0.1) node{$\theta_2$};
\draw (4.8,0.1) node{$\theta_1$};
  \end{tikzpicture}

    \end{column}
  \end{columns}

\end{frame}



\begin{frame}
  \frametitle{Multi-horizon investment}

  Dynamic multi-horizon investment is concerned with the \alert{changing capacities of investments} in the energy system over many years or even \alert{decades}.


  At which point in time should we invest in renewables/gas/storage?

  We consider several time horizons, typically years, in which plants can be dismantled or built.

  Why are we concerned with changes over decades?

  \pause

  Since many aspects of the energy system change over decades, e.g.:
  \begin{itemize}
  \item \alert{Energy consumption} (particularly in developing countries)
  \item \alert{Resource scarcity} (scarcity of oil, cobalt, rare earth metals, etc.)
  \item \alert{Political targets} (e.g. reduction of greenhouse gas emissions)
  \item \alert{Technology maturity, costs and other parameters} (e.g. efficiency)
  \item \alert{Economic growth}
  \item \alert{Behavioural change} (car sharing, less flying, online gaming, etc.)
  \end{itemize}
\end{frame}

\begin{frame}
  \frametitle{Example: political targets}

  \includegraphics[width=14cm]{co2_targets.png}

  \source{Agora Energiewende}

\end{frame}


\begin{frame}
  \frametitle{Example: Net-Zero Emissions by 2050}

  Paris-compliant 1.5$^\circ$~C scenarios from European Commission - \alert{net-zero GHG in EU by 2050}

  \vspace{.4cm}

  %Figure 6 from https://ec.europa.eu/clima/sites/clima/files/docs/pages/com_2018_733_en.pdf
  \includegraphics[width=14cm]{eu-lts-net-zero.png}


    \source{\href{https://ec.europa.eu/clima/sites/clima/files/docs/pages/com_2018_733_en.pdf}{European Commission `Clean Planet for All', 2018}}
\end{frame}


\begin{frame}
  \frametitle{Example: Cost Developments of Renewable Energy}

  LCOE = \alert{Levelised Cost of Energy} = Total Costs $/$ Energy Output
  \centering
  \includegraphics[width=14cm]{lazard-cropped.jpg}

  \source{\href{https://www.lazard.com/perspective/levelized-cost-of-energy-2017/}{Lazard's LCOE Analysis V11}}

\end{frame}


\section{Multi-horizon investment: Theoretical formulation}


\begin{frame}
  \frametitle{Discounted Total Costs}

  We will consider the total costs over \alert{multiple years} $a = 1,\dots A$.


  How do we compare costs in 2020 to those in 2040?

  \pause

  The totals costs are expressed in their
  \alert{present value} using the
 \alert{discount rate} $r$, to allow comparison between different years.

 For costs (or income) in year $a$ we \alert{discount} the costs with a factor
  \begin{equation*}
    \frac{1}{(1+r)^a}
  \end{equation*}
  because we could have invested until this year $a$ with return $r$.

  Costs in the future are \alert{worth less} from today's point of view.

  For rate $r$ we optimised the \alert{discounted total costs}
  \begin{equation*}
     \sum_{a=1}^A \frac{1}{(1+r)^a} \left\{ \textrm{Total costs in year } a \right\}
  \end{equation*}

\end{frame}



\begin{frame}
  \frametitle{Warning: Discounting over long time periods}

  Over long time periods the discounting can have a very large effect....

  \begin{columns}[T]
\begin{column}{9cm}
  \centering
  \includegraphics[width=9.5cm]{discounting.pdf}
\end{column}
\begin{column}{5cm}
  \begin{itemize}
  \item Long-term benefits aren't seen, e.g. long production life of nuclear power plants or benefits of long-lived efficiency measures
  \item Long-term costs are also suppressed, e.g. decommissioning, waste disposal, climate damages
  \item This is a \alert{controversial topic!}
  \end{itemize}
\end{column}
  \end{columns}

\end{frame}


\begin{frame}
  \frametitle{Example of Electricity System until 2050}

We optimise the discounted total costs over 30 years from 2021 to 2050
  \begin{equation*}
              \min_{\{g_{s,t,a},Q_{s,a},G_{s,a}\}} \sum_{a=1}^A \frac{1}{(1+r)^a} \left\{ \sum_{s,t} o_{s,a} g_{s,t,a} + \sum_{s,b} c_{s,b} Q_{s,b} \mathbb{I}(a  \geq b) \mathbb{I}(a < b+L_s)
 \right\}
  \end{equation*}
  Here $Q_{s,a}$ is the new capacity built in year $a$, $G_{s,a}$ is the total capacity available in year $a$, $L_s$ is the lifetime  and $\mathbb{I}$ is an indicator function that is $1$ if the condition is fulfilled, $0$ otherwise. $Q_{s,a}$ may also have fixed values for $a<1$ to represent existing capacity. $Q_{s,a}$ and $G_{s,a}$ are related by
  \begin{equation*}
    G_{s,a} = \sum_{b = 1}^{L_s} Q_{s,a-b}
  \end{equation*}

  The old constraints apply for each year $a$
  \begin{align*}
    \sum_s g_{s,t,a} & = d_{s,a}\hspace{1cm} \leftrightarrow \hspace{1cm} \l_{t,a} \\
        g_{s,t,a}   & \leq  G_{s,a}  \hspace{1cm}\leftrightarrow\hspace{1cm} \bar{\m}_{s,t,a} \\
    - g_{s,t,a}  & \leq  0  \hspace{1cm}\leftrightarrow\hspace{1cm} \ubar{\m}_{s,t,a}
  \end{align*}

\end{frame}


\begin{frame}
  \frametitle{Global constraints}

  With a long-term perspective we can now set exciting constraints.

  For example, we can restrict total \alert{emissions} over the period:
  \begin{equation*}
    \sum_{s,t,a} e_i g_{s,t,a}\leq \textrm{CAP}_{\textrm{CO}_2}
  \end{equation*}
  where $e_s$ is the specific emissions of technology $s$ (tonnes of CO$_2$ per MWh$_{\textrm{el}}$).

  Or limit \alert{resource consumption} for a technology $s$:
  \begin{equation*}
    \sum_{t,a} g_{s,t,a}\leq \textrm{CAP}_{s}
  \end{equation*}
\end{frame}


\begin{frame}
  \frametitle{Learning effects}

  Technology costs sink with accumulated manufacturing experience, particularly for new immature technologies.

  We promote $c_{s,a}$ to an optimisation variable that depends on the cumulative generator capacity.

 A simple \alert{one-factor learning model} for the costs is
  \begin{equation*}
    c_{s,a} = c_{s,0} \left(\sum_{b=1}^a Q_{s,b} \right)^{-\gamma_s}
  \end{equation*}
  where $c_{s,0}$ is the initial cost, $Q_{s,b}$ is the capacity produced in year $b$ and $\gamma_s$ is the \alert{learning parameter}.

  The \alert{learning rate} $LR$ is the reduction in cost for every doubling of production
  \begin{equation*}
    LR_s = 1-2^{-\gamma_s}
  \end{equation*}

  Example for photovoltaics: $\gamma = 0.33$ $\implies$ if cumulative production doubles, the costs reduce by 20\% (\alert{Swanson's Law}).

\end{frame}


\begin{frame}
  \frametitle{Swanson's Law for photovoltaic modules}

  The underlying dynamic is a fast decay in costs with deployment (\alert{learning-by-doing}).

  \vspace{.2cm}

  \centering
      \includegraphics[width=9.5cm]{1024px-Swansons-law.png}
\end{frame}


\begin{frame}
  \frametitle{More complicated learning models}

  In the literature there are more sophisticated learning models than the one-factor model, e.g.
  \begin{itemize}
  \item \alert{Multi-component learning models}: different parts of the
    cost experience different learning rates, e.g. some parts of the
    cost do not experience learning, such as fixed material and labour
    costs, call it $c_{s,\textrm{base}}$. Only the remainder
    experiences learning:
  \begin{equation*}
    c_{s,a} = c_{s,\textrm{base}} + (c_{s,0}-c_{s,\textrm{base}}) \left(\sum_{b=1}^a Q_{s,b} \right)^{-\gamma_s}
  \end{equation*}
  In the case of PV, $c_{s,\textrm{base}}$ would include e.g. the labour costs of installation.
  \item \alert{Multi-factor learning models}: the cost depends not just on the cumulative capacity, but on other factors such as knowledge stock $KS$ through research and development
  \begin{equation*}
    c_{s,a} = c_{s,0}\left(\sum_{b=1}^a Q_{s,b} \right)^{-\gamma_{s,1}} \left(\sum_{b=1}^a KS_{s,b} \right)^{-\gamma_{s,2}}
  \end{equation*}


  \end{itemize}

\end{frame}

\section{Multi-horizon investment: Simplified example}



\begin{frame}
  \frametitle{Simplified example}

  \url{https://nworbmot.org/courses/esm-2020/lectures/notebooks/dynamic_investment.ipynb}

  Time period: 2021 until 2070. Discount rate: $r = 0.05$.

  Constant electricity demand $d_{t,a} = d = 100$ GW.

  At the start of the simulation there is already 100~GW of 20-year-old coal plants.

  3 generation technologies are available that are dispatchable (for Concentrating Solar Power (CSP) need good direct solar insolation, e.g. New Mexico or Morocco).


  \ra{1.1}
  \begin{table}[!t]
    \begin{tabular}{lrrrrrr}
      \toprule
      Tech & Capital costs & Marg. costs & LCOE & Cap & Emissions & Lifetime    \\
       &  (\euro MW$^{-1}$ a$^{-1}$) &  (\euro MWh$_{\textrm{el}}^{-1}$) &  (\euro MWh$_{\textrm{el}}^{-1}$) & factor & (tCO$_2$MWh$_{\textrm{el}}^{-1}$) & years \\
      \midrule
      Coal &30*8760 & 20 & 50 & 1 & 1 & 40\\
      Nuclear & 65*8760 & 10 & 75 & 1 & 0 & 40 \\
      CSP & 150*8760 & 0 & 150 & 1 & 0 & 30 \\
      \bottomrule
    \end{tabular}
  \end{table}



\end{frame}

\begin{frame}
  \frametitle{Simplified example}

  Since each technology can generate continuously and the demand is constant, we assume $g_{s,t,a}$ is constant for all $t$
   \begin{equation*}
    g_{s,t,a} = g_{s,a} \leq G_{s,a}
   \end{equation*}

  This simplifies the optimisation problem considerably:
  \begin{equation*}
              \min_{\{g_{s,t,a},Q_{s,a},G_{s,a}\}} \sum_{a=1}^A \frac{1}{(1+r)^a} \left\{ \sum_{s} o_{s,a} g_{s,a}\cdot 8760 + \sum_{s,b} c_{s,b} Q_{s,b} \mathbb{I}(a  \geq b) \mathbb{I}(a < b+L_s)
 \right\}
  \end{equation*}
  with constraints for each year $a$
  \begin{align*}
    \sum_s g_{s,a} & = d
  \end{align*}
\end{frame}


\begin{frame}
  \frametitle{Vanilla Version: No CO$_2$ budget, no learning, no discounting}

  Only new coal is built, since it's cheapest.

  Total costs without discounting: 50\euro/MWh $\cdot$ 8760 $\cdot$ 100 GW $\cdot$ 50 years = 2190 billion \euro

    \centering
  \includegraphics[width=11cm]{no_co2-no_learning-dispatch.pdf}
\end{frame}


\begin{frame}
  \frametitle{Vanilla Version: No CO$_2$ budget, no learning, discounting}

  Only coal is built, since it's cheapest.

  Total costs with discount rate 5\%: 840 billion \euro

   \centering
  \includegraphics[width=11cm]{no_co2-no_learning-dispatch.pdf}

\end{frame}




\begin{frame}
  \frametitle{CO$_2$ budget, no learning, discounting}

  Limit CO$_2$ to 20\% of coal emissions. Nuclear takes over before coal lifetimes are finished. Why is it built only later in the period (even when no existing plants assumed)? (Hint: discounting)

  Total costs with discount rate 5\%: 1147 billion \euro

  \centering
  \includegraphics[width=10cm]{co2-0p2-no_learning-dispatch.pdf}

\end{frame}



\begin{frame}
  \frametitle{CO$_2$ budget, learning for CSP, discounting}


  Limit CO$_2$ to 20\% of coal emissions. CSP has learning rate 20\%, $\gamma = 0.33$, and a base long-term potential LCOE of 35~\euro/MWh that represents material and labour costs.

  Total costs with discount rate 5\%: 1020 billion \euro

  \centering
  \includegraphics[width=10cm]{co2-0p2-learning-dispatch.pdf}

\end{frame}


\begin{frame}
  \frametitle{CO$_2$ budget, learning for CSP, discounting}

  LCOE needs subsidy initially to push down learning curve, since it is more expensive than incumbent technologies. But from 2034 onwards it is the most competitive technology.

  \centering
  \includegraphics[width=11cm]{co2-0p2-learning-lcoe.pdf}

\end{frame}

\begin{frame}
  \frametitle{Lessons from this example}

  \begin{itemize}
  \item Non-linear effects such as learning-by-doing make the results hard to predict
  \item It may be cost-effective in the long-run to subsidise technologies that are uncompetitive today
  \item Depending on how subsidy and policy is arranged, there could be \alert{path dependencies}
  \end{itemize}

  To improve the realism of this example we need to:
  \begin{itemize}
  \item Include more technologies, spatial resolution
  \item Consider more representative times per year to capture the variability of renewables and load
  \end{itemize}

\end{frame}

\end{document}
