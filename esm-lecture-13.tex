% NB: use pdflatex to compile NOT pdftex.  Also make sure youngtab is
% there...

% converting eps graphics to pdf with ps2pdf generates way too much
% whitespace in the resulting pdf, so crop with pdfcrop
% cf. http://www.cora.nwra.com/~stockwel/rgspages/pdftips/pdftips.shtml




\documentclass[10pt,aspectratio=169,dvipsnames]{beamer}


\usetheme[color/block=transparent]{metropolis}

\usepackage[absolute,overlay]{textpos}
\usepackage{booktabs}

\usepackage{graphbox} %loads graphicx package

\usepackage{pbox}

\usepackage{adjustbox}

\usepackage[utf8]{inputenc}


\usepackage[scale=2]{ccicons}


\usepackage[official]{eurosym}


\usepackage{tikz}
\usetikzlibrary{arrows.meta}


\usepackage[europeanresistors,americaninductors]{circuitikz}



\setbeamerfont{alerted text}{series=\bfseries}
%\setbeamercolor{alerted text}{fg=BurntOrange}
%\setbeamercolor{alerted text}{fg=mDarkTeal}
%\setbeamercolor{alerted text}{fg=Brown}  % Goran
\setbeamercolor{alerted text}{fg=Mahogany}  % Goran

\setbeamercolor{background canvas}{bg=white}


\newcommand{\R}{\mathbb{R}}

\def\l{\lambda}
\def\m{\mu}
\def\d{\partial}
\def\cL{\mathcal{L}}
\def\co2{CO${}_2$}


\def\el{${}_{el}$}
\def\th{${}_{th}$}
\def\gas{${}_{gas}$}



\newif\ifgoat

\goattrue                       %



\newcommand*\rot{\rotatebox{90}}


\usepackage{pifont}
\newcommand*\OK{\ding{51}}


%use this to add space between rows
\newcommand{\ra}[1]{\renewcommand{\arraystretch}{#1}}

% for sources http://tex.stackexchange.com/questions/48473/best-way-to-give-sources-of-images-used-in-a-beamer-presentation

\setbeamercolor{framesource}{fg=gray}
\setbeamerfont{framesource}{size=\tiny}


\newcommand{\source}[1]{\begin{textblock*}{5cm}(10.5cm,8.35cm)
    \begin{beamercolorbox}[ht=0.5cm,right]{framesource}
        \usebeamerfont{framesource}\usebeamercolor[fg]{framesource} Source: {#1}
    \end{beamercolorbox}
\end{textblock*}}

\usepackage{hyperref}


%\usepackage[pdftex]{graphicx}


\graphicspath{{graphics/}}

\DeclareGraphicsExtensions{.pdf,.jpeg,.png,.jpg}



\def\goat#1{{\scriptsize\color{green}{[#1]}}}



\let\olditem\item
\renewcommand{\item}{%
\olditem\vspace{5pt}}

\title{Energy System Modelling\\ Summer Semester 2019, Lecture 11}
%\subtitle{---}
\author{
  {\bf Dr. Tom Brown}, \href{mailto:tom.brown@kit.edu}{tom.brown@kit.edu}, \url{https://nworbmot.org/}\\
  \emph{Karlsruhe Institute of Technology (KIT), Institute for Automation and Applied Informatics (IAI)}
}

\date{\vspace{.3cm}14th June 2019}


\titlegraphic{%
  \vspace{6cm}
  \hspace{10cm}
    \includegraphics[trim=0 0cm 0 0cm,height=1.8cm,clip=true]{kit.png}
}

\begin{document}

\maketitle


\begin{frame}

  \frametitle{Table of Contents}
  \setbeamertemplate{section in toc}[sections numbered]
  \tableofcontents[hideallsubsections]
\end{frame}



\section{Effect of spatial scale on results of energy system optimisations}




\begin{frame}{Motivation: Transmission bottlenecks}

  Many of the results we've examined so far have aggregated countries
  to a single node. However, there are also transmission network
  bottlenecks \alert{within} countries (e.g. North to South Germany).


\centering
\includegraphics[width=8cm]{europe-transmission}
\source{ENTSO-E}


\end{frame}




\begin{frame}{Motivation: Wind and solar resource variation}

  There is also considerable variation in wind and solar resources...

    \begin{columns}[T]
\begin{column}{5cm}
  \includegraphics[trim=0 0cm 0 0cm,width=4.7cm,clip=true]{SolarGIS-Solar-map-Germany-de}
\end{column}
\begin{column}{6cm}
  \vspace{.3cm}
  \includegraphics[trim=0 0cm 0 0cm,width=6cm,clip=true]{43_Mittlere_Windgeschwindigkeit_100_m_Deutschland}
\end{column}
    \end{columns}

\end{frame}





\begin{frame}
  \frametitle{Spatial resolution}

\begin{columns}
\begin{column}{7cm}
We need spatial resolution to:
\begin{itemize}
\item capture the \alert{geographical variation} of renewables resources and the load
\item capture \alert{spatio-temporal effects} (e.g. size of wind correlations across the continent)
\item represent important \alert{transmission constraints}
\end{itemize}

BUT we do not want to have to model all 5,000 network nodes of the European system.
\end{column}
\begin{column}{6cm}
\centering
% No, we really do not not want to optimise a model of 4653 substation, 5613 AC lines and 26 DC lines of the European HV electricity network.
\includegraphics[width=6.5cm]{pre-2-network-full.pdf}
\source{Own representation of Bart Wiegman's GridKit extract of the
  online ENTSO-E map, \url{https://doi.org/10.5281/zenodo.55853}}
\end{column}
\end{columns}
\end{frame}

\begin{frame}
  \frametitle{Clustering: Many algorithms in the literature}

  There are lots of algorithms for clustering networks, particularly in the engineering literature:
  \begin{itemize}
    \item $k$-means clustering on (electrical) distance
    \item $k$-means on load distribution
    \item Community clustering (e.g. Louvain)
    \item Spectral analysis of Laplacian matrix
    \item Clustering of Locational Marginal Prices with nodal pricing (sees congestion and RE generation)
    \item PTDF clustering
    \item Cluster nodes with correlated RE time series
  \end{itemize}

  The algorithms all serve different purposes (e.g. reducing part of
  the network on the boundary, to focus on another part).

  Not always tested on real network data.
\end{frame}

\begin{frame}
  \frametitle{$k$-means clustering on load \& conventional generation}

  Our \alert{goal}: maintain main transmission corridors of today to
  investigate highly renewable scenarios with no grid expansion. Since generation fleet is totally rebuilt, do not want to rely on current generation dispatch (like e.g. LMP algorithm).

  Today's grid was laid out to connect big generators and load centres.

  \alert{Solution}:
  Cluster nodes based on load and conventional generation capacity using
  $k$-means.

  I.e. find $k$ centroids and the corresponding $k$-partition of the
  original nodes that minimises the sum of squared distances from each
  centroid to its nodal members:
  \begin{equation}
    \min_{\{x_c\}} \sum_{c=1}^k \sum_{n \in N_c} w_n || x_c - x_n ||^2
  \end{equation}
  where each node is weighted $w_n$ by the average load and the
  average conventional generation there.


\end{frame}



\begin{frame}
  \frametitle{Reconstitution of network}

  Once the partition of nodes is determined:
  \begin{itemize}
  \item A new node is created to represent each set of clustered nodes
  \item Hydro capacities and load is aggregated at the node; VRE (wind and solar) time series are aggregated, weighted by capacity factor; potentials for VRE aggregated
  \item Lines between clusters replaced by single line with length 1.25 $\times$ crow-flies-distance, capacity and impedance according to replaced lines
    \item $n-1$ blanket safety margin factor grows from 0.3 with $\geq 200$ nodes to 0.5 with 37 nodes (to account for aggregation)
  \end{itemize}

\end{frame}

\begin{frame}
  \frametitle{$k$-means clustering: Networks}

\centering
 \includegraphics[width=13cm]{networks-clustered-crop}
  % $\vcenter{\hbox{\includegraphics[width=4cm]{pre-2-network-full}}}$
  % $\vcenter{\hbox{\includegraphics[width=4cm]{pre-2-network-362-LV-1}}}$
  % $\vcenter{\hbox{\includegraphics[width=4cm]{pre-2-network-181-LV-1}}}$ \\
  % $\vcenter{\hbox{\includegraphics[width=4cm]{pre-2-network-128-LV-1}}}$
  % $\vcenter{\hbox{\includegraphics[width=4cm]{pre-2-network-64-LV-1}}}$
  % $\vcenter{\hbox{\includegraphics[width=4cm]{pre-2-network-37-LV-1}}}$



\end{frame}



\begin{frame}
  \frametitle{Question of spatial resolution}

  How is the overall minimum of the cost objective (building and
  running the electricity system) affected by an increase of spatial
  resolution in each country?

  We expect
  \begin{itemize}
  \item A better representation of existing internal bottlenecks will prevent the transport of e.g. offshore wind to the South of Germany.
  \item Localised areas of e.g. good wind can be better exploited by the optimisation.
  \end{itemize}

  Which effect will win?

  First we only optimize the gas, wind and solar generation
  capacities, the long-term and short-term storage capacities and
  their economic dispatch including the available hydro facilities
  \alert{without grid expansion}.
\end{frame}



\begin{frame}
  \frametitle{Nodal energy shares per technology (w/o grid expansion)}

\includegraphics[width=15cm]{legend-flat}
\begin{columns}[T]
\begin{column}{8cm}


  \includegraphics[trim=0 0cm 0 2cm,width=8cm]{euro-pie-pre-7-branch_limit-1-37.png}

\end{column}
\begin{column}{8cm}

  \raisebox{-6.6cm}{

  \includegraphics[trim=0 0cm 0 2cm,width=8cm]{euro-pie-pre-7-branch_limit-1-256.png}
  }

\end{column}

\end{columns}


\end{frame}

\begin{frame}
   \frametitle{Costs: System cost w/o grid expansion}


\begin{columns}[T]
\begin{column}{8.5cm}
  \includegraphics[width=9.5cm]{cost_eu_1}
\end{column}
\begin{column}{6cm}

  \vspace{2cm}

  \begin{itemize}
  \item Steady total system cost at \euro~260 billion per year
   \item This translates to \euro~82/MWh (compared to today of \euro~50/MWh to \euro~60/MWh)
  \end{itemize}
\end{column}
\end{columns}

\end{frame}


\begin{frame}
  \frametitle{Costs: System cost and break-down into technologies {\large (w/o grid expansion)}}
\begin{columns}[T]
\begin{column}{8.5cm}
  \includegraphics[width=8.5cm]{cost_eu_breakdown_by_cap-1}
\end{column}
\begin{column}{6cm}

  If we break this down into technologies:
  \begin{itemize}
  \item 37 clusters captures around half of total network volume
  \item Redistribution of capacities from offshore wind to solar
  \item Increasing solar share is accompanied by an increase of
    battery storage
  \item Single countries do not stay so stable
  \end{itemize}
\end{column}
\end{columns}
\end{frame}

\begin{frame}
  \frametitle{Costs: Focus on Germany (w/o grid expansion)}

\begin{columns}[T]
\begin{column}{8.5cm}
\centering
  \includegraphics[width=8.5cm]{cost_de_breakdown-1-0}
  %\includegraphics[width=6cm]{pre-2-costs_de_storage}
\end{column}
\begin{column}{6cm}
  \begin{itemize}
  \item Offshore wind replaced by onshore wind at better sites and solar (plus batteries), since
  the represented transmission bottlenecks make it impossible to
  transport the wind energy away from the coast
  \item the effective onshore wind capacity factors increase from $26\%$ to up to $42\%$
  \item Investments stable at 181 clusters and above
  \end{itemize}
\end{column}
\end{columns}


\end{frame}




\begin{frame}
  \frametitle{Interaction between network expansion and spatial scale}

  6 different scenarios of network expansion by constraining the
  overall transmission line volume in relation to today's line volume $\mathrm{CAP}_{\mathrm{trans}}^{\mathrm{today}}$, given length $d_\ell$ and capacity
  $F_\ell$ of each line $\ell$:
  \begin{align}
    F_\ell & \geq F_\ell^{\mathrm{today}} \\
    \sum_\ell d_\ell F_\ell & \leq \mathrm{CAP}_{\mathrm{trans}}
  \end{align}
  where
  \begin{equation}
    \mathrm{CAP}_{\mathrm{trans}} = x \, \mathrm{CAP}_{\mathrm{trans}}^{\mathrm{today}}
  \end{equation}

  for $x = 1$ (today's grid) $x = 1.125,1.25,1.5,2$, $x=3$ (optimal for overhead line at high number of cluster).
\end{frame}





\begin{frame}
  \frametitle{With expansion}

  \includegraphics[width=15cm]{legend-flat}
  \begin{columns}[T]

\begin{column}{8cm}

  \includegraphics[trim=0 0cm 0 2cm,width=8cm]{euro-pie-pre-7-branch_limit-1-5-256.png}

\end{column}
\begin{column}{8cm}

  \raisebox{-6.6cm}{

  \includegraphics[trim=0 0cm 0 2cm,width=8cm]{euro-pie-pre-7-branch_limit-3-256.png}
  }

\end{column}

\end{columns}


\end{frame}



\begin{frame}
  \frametitle{Costs: Total system cost}

\centering

\begin{columns}[T]
\begin{column}{8.5cm}
  \includegraphics[width=9.5cm]{cost_eu}

\end{column}
\begin{column}{7cm}

  \begin{itemize}
  \item Steady cost for No Expansion (1)
  \item For expansion scenarios, as clusters increase, the better expoitation of good sites decreases costs faster than transmission bottlenecks increase them
  \item Decrease in cost is v. non-linear as grid expanded (25\% grid expansion gives 50\% of optimal cost reduction)
  \item Only a moderate $20-25\%$ increase in costs from the Optimal Expansion
    scenario (3) to the No Expansion scenario (1).
  \end{itemize}

\end{column}

\end{columns}

\end{frame}


\begin{frame}
  \frametitle{Costs: Break-down into technologies}

\centering

  \includegraphics[width=12cm]{cost_eu_breakdown_by_cap}
\end{frame}



\begin{frame}
  \frametitle{Costs: Focus on Germany (CAP = 3)}

\begin{columns}[T]
\begin{column}{8.5cm}
\centering
  \includegraphics[width=8.5cm]{cost_de_breakdown-3-0}
  %\includegraphics[width=6cm]{pre-2-costs_de_storage}
\end{column}
\begin{column}{6cm}
  \begin{itemize}
  \item Investment reasonably stable at 128 clusters and above
  \item System consistently dominated by wind
  \item No solar or battery for any number of clusters
  \end{itemize}
\end{column}
\end{columns}


\end{frame}




\begin{frame}
  \frametitle{Behaviour as CAP is changed}


\begin{columns}[T]
\begin{column}{8.5cm}
\centering
  \includegraphics[width=8.5cm]{costs_per_cluster_181}
  %\includegraphics[width=6cm]{pre-2-costs_de_storage}
\end{column}
\begin{column}{6cm}
  \begin{itemize}
%\item   %Big reduction in curtailment
    \item Same non-linear development with high number of nodes that we saw with one node per country
      \item Most of cost reduction happens with small expansion; cost
        rather flat once capacity has doubled, reaching minimum (for
        overhead lines) at 3 times today's capacities
      \item Solar and batteries decrease significantly as grid expanded
  \item  Reduction in storage losses too
  \end{itemize}
\end{column}
\end{columns}
\end{frame}



\begin{frame}
  \frametitle{Locational Marginal Prices CAP=1 versus CAP=3}



  \begin{columns}[T]

\begin{column}{8cm}

  \hspace{1cm}  With today's capacities:

  \includegraphics[width=8cm]{lmp-1-256.pdf}

\end{column}
\begin{column}{8cm}

  \hspace{1cm}With three times today's grid:

  \includegraphics[width=8cm]{lmp-3-256.pdf}

\end{column}

\end{columns}


\end{frame}




\begin{frame}
  \frametitle{Grid expansion CAP shadow price for 181 nodes as CAP relaxed}


\begin{columns}[T]
\begin{column}{10cm}
\centering
  \includegraphics[width=11cm]{shadows-branch_limit}
  %\includegraphics[width=6cm]{pre-2-costs_de_storage}
\end{column}
\begin{column}{4cm}
  \begin{itemize}
%\item   %Big reduction in curtailment
  \item With overhead lines the optimal system has around 3 times today's transmission volume
  \item With underground cables (5-8 times more expensive) the optimal system has around 1.3 to 1.6 times today's transmission volume
  \end{itemize}
\end{column}
\end{columns}

\end{frame}


\begin{frame}
  \frametitle{CO2 prices versus line cap for 181 clusters}



\begin{columns}[T]
\begin{column}{10cm}
\centering
  \includegraphics[width=11cm]{shadows-co2}
  %\includegraphics[width=6cm]{pre-2-costs_de_storage}
\end{column}
\begin{column}{4cm}
  \begin{itemize}
    \vspace{2cm}
%\item   %Big reduction in curtailment
  \item CO2 price of between 150 and 250 \euro/tCO2 required to reach these solutions, depending on line volume cap
  \end{itemize}
\end{column}
\end{columns}

\end{frame}




% \begin{frame}
%   \frametitle{Comparison of results: energy}

% \centering

%   \includegraphics[width=4.5cm]{pre-2-clusters-yearly_energy-LV-inf}
%   \includegraphics[width=4.5cm]{pre-2-clusters-yearly_energy-LV-4}
%   \includegraphics[width=4.5cm]{pre-2-clusters-yearly_energy-LV-1}

% \end{frame}


% \section{Conclusions}




\begin{frame}
  \frametitle{Conclusions}

  \begin{itemize}
    \item This is \alert{no single solution} for highly renewable systems, but a \alert{family of solutions}  with different costs and compromises
    \item Generation costs always dominate grid costs, but the grid can cause higher generation costs if expansion is restricted
    \item Systems with no grid extension beyond today are up to 25\% more expensive, but small grid extensions (e.g. 25\% more capacity than today) can lock in big savings
    \item Need at least around 200 clusters for Europe to see grid bottlenecks if no expansion
    \item Can get away with $\sim 120$ clusters for Europe if grid expansion is allowed
      \item Much of the stationary storage needs can be eliminated by sector-coupling: DSM with electric vehicles, thermal storage; this makes grid expansion less beneficial
    \item Understanding the need for \alert{flexibility at different temporal and spatial scales} is key to mastering the complex interactions in the energy system
  \end{itemize}
\end{frame}

\section{Algorithms to accelerate computations}



\begin{frame}
  \frametitle{Cycle formulation of linear power flow}

  We can use dual graph theory to decompose the flows in the network into two parts:
  \begin{enumerate}
  \item A flow on a spanning tree of the network, uniquely determined by nodal $\mathbf{p}$ (ensuring KCL)
  \item Cycle flows, which don't affect KCL; their strength is fixed by enforcing KVL
  \end{enumerate}

\begin{circuitikz}
  \draw
  (0,3)
  to [short,i^=$f_1$,*-*] (3,3)
  to [short,i>=$f_2$,*-*] (3,0)
  to [short,i>=$f_3$,*-*] (0,0)
  to [short,i>=$f_4$,*-*] (0,3);
  \draw (3,3)   to [short,i^=$f_5$,*-*] (0,0);

  \draw (1.5,-1.2) node{$f_\ell$};


  \draw (4,1.5) node{$=$};

  \draw (4,-1.2) node{$=$};


  \draw[red]
  (5,3)
  to [short,i^=$t_1$,*-*] (8,3)
  to [short,i>=$t_2$,*-*] (8,0)
  to [short,i>=$t_3$,*-*] (5,0);

  \draw (6.5,-1.2) node{$t_\ell$};


  \draw (9,1.5) node{$+$};

  \draw (9,-1.2) node{$+$};



  \draw[blue]
  (10,3)
  to [short,i>=$$,*-*] (12.8,3)
  to [short,i>=$$,*-*] (10,0.2)
  to [short,i>=$c_1$,*-*] (10,3);

  \draw[blue]
  (13,0)
  to [short,i>=$$,*-*] (10.2,0)
  to [short,i>=$$,*-*] (13,2.8)
  to [short,i>=$c_2$,*-*] (13,0);


  \draw (11.5,-1.2) node{$\sum_k C_{\ell,k} c_k$};


\end{circuitikz}


\end{frame}


\begin{frame}
  \frametitle{Cycle formulation of linear power flow}

  The $N-1$ tree flows $\mathbf{t}$ are determined directly from the
  $N$ nodal powers $p_n$ and the network power balance constraint
  $\sum_n p_n = 0$.

  We solve for the $L-N+1$ cycle flows $c_k$ by enforcing  the $L-N+1$ KVL equations:
  \begin{equation*}
    C^t X \mathbf{f} = C^t X (\mathbf{t} + C \mathbf{c}) = 0
  \end{equation*}

  The matrix $C$ is the incidence matrix of the \alert{weak dual
    graph}, $C^t X C$ is the weighted Laplacian of the dual graph and
  the above equation becomes a discrete Poisson equation:
  \begin{equation*}
    C^t X C \mathbf{c} =  - C^t X \mathbf{t}
  \end{equation*}

\end{frame}



\begin{frame}
  \frametitle{LOPF speedup with cycle flows}
\begin{columns}[T]
  \begin{column}{7cm}
  \includegraphics[width=7.5cm]{lopf-ne-speedup-per-nodes-mean-with-ci-99-regression}
  \end{column}
  \begin{column}{6cm}
    \vspace{0.5cm}
    Using \alert{cycle flows instead of voltage angles} we found for generation expansion optimisation (fixed grid):
    \begin{itemize}
    \item A speed-up of up to \alert{200 times}
    \item Average speed-up of \alert{factor 12}
    \item Speed-up is highest for \alert{large networks with lots of renewables}
    \end{itemize}
  \end{column}

\end{columns}

\vspace{.3cm}

  \footnotesize
H.~Ronellenfitsch, D.~Manik, J.~Hörsch, {\bf T.~Brown, D.~Witthaut}, \emph{``Dual theory of transmission line outages,''} 2017, IEEE Transactions on Power Systems

J.~Hörsch, H.~Ronellenfitsch, {\bf D.~Witthaut, T.~Brown}, \emph{``Linear Optimal Power Flow Using Cycle Flows,''} 2017

\end{frame}



\begin{frame}{Copyright}


  Unless otherwise stated, the graphics and text are Copyright \copyright Tom Brown, 2017.

  The source \LaTeX, self-made graphics and Python code used to
  generate the self-made graphics are available here:

  \url{http://nworbmot.org/talks.html}

  The graphics and text for which no other attribution are given are licensed under a
  \href{http://creativecommons.org/licenses/by-sa/4.0/}{Creative Commons
  Attribution-ShareAlike 4.0 International License}.

  \begin{center}\ccbysa\end{center}

\end{frame}




\begin{frame}
  \frametitle{Cost and other assumptions}

  \begin{table}
\centering
\begin{tabular}{@{}lrlrr@{}}
\toprule
Quantity                &Overnight  Cost [\euro]  &Unit & FOM [\%/a] & Lifetime [a] \\
\midrule
Wind onshore    &1182   &kW\el &3 & 20   \\
Wind offshore  &2506   &kW\el  &3& 20  \\
Solar PV           &600   &kW\el &4 & 20   \\
Gas             &400    &kW\el  &4& 30  \\
%Gas marginal            &75     &\euro{}/MWh$_{\textrm{e}}$    \\
Battery storage         &1275   &kW\el  & 3 & 20 \\
Hydrogen storage        &2070   &kW\el  & 1.7 &20 \\
Transmission line       &400    &MWkm & 2 & 40\\
\bottomrule
\end{tabular}
\end{table}
  Interest rate of 7\%, storage efficiency losses, only gas has \co2 emissions, gas marginal costs.
\end{frame}


\begin{frame}
  \frametitle{Shadow costs of line extension CAP for 3 times today's volume}



\begin{columns}[T]
\begin{column}{8.5cm}
\centering
  \includegraphics[width=8.5cm]{shadows}
  %\includegraphics[width=6cm]{pre-2-costs_de_storage}
\end{column}
\begin{column}{6cm}
  \begin{itemize}
%\item   %Big reduction in curtailment
  \item For 200+ nodes the shadow price converges on the annual cost of a MWkm of overhead line (around \euro~30/a/MWkm)
  \item Value of lines is much higher with smaller number of clusters. Why?
\item Possible reasons: inter-connectors in general weaker than country-internal connectors;  more nodes means more flexibility to avoid network expansion
  \end{itemize}
\end{column}
\end{columns}
\end{frame}


\end{document}
