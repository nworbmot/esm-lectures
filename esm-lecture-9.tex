% NB: use pdflatex to compile NOT pdftex.  Also make sure youngtab is
% there...

% converting eps graphics to pdf with ps2pdf generates way too much
% whitespace in the resulting pdf, so crop with pdfcrop
% cf. http://www.cora.nwra.com/~stockwel/rgspages/pdftips/pdftips.shtml




\documentclass[10pt,aspectratio=169,dvipsnames]{beamer}

\usetheme[color/block=transparent]{metropolis}

\usepackage[absolute,overlay]{textpos}
\usepackage{booktabs}
\usepackage[utf8]{inputenc}

\usepackage{tikz}


\usepackage[scale=2]{ccicons}

\usepackage[official]{eurosym}


%use this to add space between rows
\newcommand{\ra}[1]{\renewcommand{\arraystretch}{#1}}


\newcommand{\R}{\mathbb{R}}


\setbeamerfont{alerted text}{series=\bfseries}
\setbeamercolor{alerted text}{fg=Mahogany}
\setbeamercolor{background canvas}{bg=white}



\def\l{\lambda}
\def\m{\mu}
\def\d{\partial}
\def\cL{\mathcal{L}}
\def\co2{CO${}_2$}
\def\bra#1{\left\langle #1\right|}
\def\ket#1{\left| #1\right\rangle}
\newcommand{\braket}[2]{\langle #1 | #2 \rangle}
\newcommand{\norm}[1]{\left\| #1 \right\|}
\def\corr#1{\Big\langle #1 \Big\rangle}
\def\corrs#1{\langle #1 \rangle}


\def\mw{\text{ MW}}
\def\mwh{\text{ MWh}}
\def\emwh{\text{ \euro/MWh}}

\newcommand{\ubar}[1]{\text{\b{$#1$}}}


% for sources http://tex.stackexchange.com/questions/48473/best-way-to-give-sources-of-images-used-in-a-beamer-presentation

\setbeamercolor{framesource}{fg=gray}
\setbeamerfont{framesource}{size=\tiny}


\newcommand{\source}[1]{\begin{textblock*}{5cm}(10.5cm,8.35cm)
    \begin{beamercolorbox}[ht=0.5cm,right]{framesource}
        \usebeamerfont{framesource}\usebeamercolor[fg]{framesource} Source: {#1}
    \end{beamercolorbox}
\end{textblock*}}

\usepackage{hyperref}

\usepackage{tikz}


\usepackage[europeanresistors,americaninductors]{circuitikz}


%\usepackage[pdftex]{graphicx}


\graphicspath{{graphics/}}

\DeclareGraphicsExtensions{.pdf,.jpeg,.png,.jpg,.gif}



\def\goat#1{{\scriptsize\color{green}{[#1]}}}



\let\olditem\item
\renewcommand{\item}{%
\olditem\vspace{5pt}}

\title{Energy System Modelling\\ Summer Semester 2020, Lecture 9}
%\subtitle{---}
\author{
  {\bf Dr. Tom Brown}, \href{mailto:tom.brown@kit.edu}{tom.brown@kit.edu}, \url{https://nworbmot.org/}\\
  \emph{Karlsruhe Institute of Technology (KIT), Institute for Automation and Applied Informatics (IAI)}
}

\date{}


\titlegraphic{
  \vspace{0cm}
  \hspace{10cm}
    \includegraphics[trim=0 0cm 0 0cm,height=1.8cm,clip=true]{kit.png}

\vspace{5.1cm}

  {\footnotesize

  Unless otherwise stated, graphics and text are Copyright \copyright Tom Brown, 2020.
  Graphics and text for which no other attribution are given are licensed under a
  \href{https://creativecommons.org/licenses/by/4.0/}{Creative Commons
  Attribution 4.0 International Licence}. \ccby}
}

\begin{document}

\maketitle


\begin{frame}

  \frametitle{Table of Contents}
  \setbeamertemplate{section in toc}[sections numbered]
  \tableofcontents[hideallsubsections]
\end{frame}


\section{Present value and discounting}


\begin{frame}
  \frametitle{The value of money depends on time}


  \alert{Question 1:} What would you prefer: \euro 1000 today, or \euro 1000 in 3 years?

  \pause

  \euro 1000 today can be invested in the bank with an interest rate of 5\%.

  After 3 years you would have
  \begin{equation*}
     1000 \cdot (1 + 0.05)^3 = 1158
  \end{equation*}

  \alert{Answer 1:} Best to take the money today and use the opportunity to invest!

  \vspace{.5cm}

 ``Money in the future is worth less than money today.''

\end{frame}


\begin{frame}
  \frametitle{The value of money depends on time}


  \alert{Question 2:} What would you prefer: \euro 1000 today, or \euro 1300 in 5  years?

  \pause

  If you invested \euro 1000 today, after 5 years you would have only
  \begin{equation*}
     1000 \cdot (1 + 0.05)^5 = 1276
  \end{equation*}

  \alert{Answer 2:} Best to wait for the \euro 1300 in 5 years!


\end{frame}



\begin{frame}
  \frametitle{Present value}

  To allow comparison between income and outgoings in different years, we need to agree on a particular point in time to evaluate the cash flows.

  The simplest and most frequently used time point: today's value, known as the \alert{present value}.

  For an \alert{interest rate} $r$ we multiply the income or outgoings in year $t$ by the \alert{discount factor}
  \begin{equation*}
    \frac{1}{(1+r)^t}
  \end{equation*}
  to calculate the present value. We have \alert{discounted} the future cash flow.

  Future income or outgoings are \alert{worth less} from today's point of view (as long as $r$ is positive).

  ``Money in the future is worth less than money today.''

\end{frame}

\begin{frame}
  \frametitle{Example: present value}

  For our example with interest rate 5\% we can now order the options:
  \ra{1.1}
  \begin{table}[!t]
    \begin{tabular}{rrr}
      \toprule
      Income (\euro) & Year & Present value (\euro)  \\
      \midrule
      1000  &  3 & $\frac{1000}{(1+0.05)^3} = 863$  \\
      1000  &  0 & $\frac{1000}{(1+0.05)^0} = 1000$ \\
      1300  &  5 & $\frac{1300}{(1+0.05)^5} = 1019$ \\
      \bottomrule
    \end{tabular}
  \end{table}

\end{frame}



\section{Investment calculations}


\begin{frame}
  \frametitle{Motivation: Power plant investment}

  A company is considering investing in a photovoltaic plant on its roof. The key figures:


  \begin{columns}[T]
\begin{column}{6cm}

    \begin{table}[!t]
    \begin{tabular}{lr}
      \toprule
      Size &  100 kW \\
      Investment cost & 800 \euro kW$^{-1}$ \\
      Operating cost & 20 \euro kW$^{-1}$ a$^{-1}$\\
      Feed-In Tariff & 0.1 \euro kWh$^{-1}$\\
      Full load hours & 1000 \\
      Period of subsidy & 20 years \\
      \bottomrule
    \end{tabular}
  \end{table}
\end{column}
\begin{column}{6cm}

    \includegraphics[width=6.5cm]{PV_Anlage_auf_der_Rottweiler_Stadthalle.jpg}
\end{column}
  \end{columns}

  \vspace{.5cm}

  The company can invest its money elsewhere for a return of 5\%.

  \alert{Is it worthwhile to invest in the photovoltaic plant?}

  \source{\href{https://www.ise.fraunhofer.de/content/dam/ise/de/documents/publications/studies/DE2018_ISE_Studie_Stromgestehungskosten_Erneuerbare_Energien.pdf}{Fraunhofer ISE Stromgestehungskosten 2018}, \href{https://de.wikipedia.org/wiki/ENRW_Energieversorgung_Rottweil\#/media/Datei:PV_Anlage_auf_der_Rottweiler_Stadthalle.jpg}{Wikipedia}}

\end{frame}

\begin{frame}
  \frametitle{Investment calculations}

  An \alert{investment calculation} quantifies the financial costs and benefits of an investment, assuming that future income and outgoings can be predicted.

  It considers
  \begin{itemize}
  \item \alert{Capital costs} - Costs for investments and installation
  \item \alert{Consumption costs} - Fuel, other materials (e.g. lubricants for wind turbine), etc.
  \item \alert{Operating costs} - Maintenance, wages, insurance, management, etc.
  \item \alert{Income} - depends on market price, subsidies, and production
  \end{itemize}

\end{frame}



\begin{frame}
  \frametitle{Dynamic investment calculation}

  For a \alert{dynamic investment calculation} we sum the present values of all income and outgoings over the $T$ years of operation taking account of the interest rate $r$ to get the \alert{Net Present Value (NPV)}:
  \begin{equation*}
    NPV = \sum_{t=0}^T \frac{ - I_t - V_t - B_t + U_t}{(1+r)^t}
  \end{equation*}
  where $I_t$ is the capital expenditure in year $t$, $V_t$ the
  consumption costs (e.g. for fuel cost $o_t$ and annual production $Q_t$, $V_t = o_t\cdot Q_t$), $B_t$ the operating costs und $U_t$ the income (e.g. average market value $\lambda_t$ times annual production
  $Q_t$, $U_t = \lambda_t\cdot Q_t$).

  \alert{Conclusion:} If $NPV > 0$, the investment is worthwhile.

  If $NPV < 0$, better to invest with a rate of return of $r$ elsewhere.

  For comparisons between different investments, a higher NPV should be preferred.
\end{frame}


\begin{frame}
  \frametitle{Example: Rooftop photovoltaic unit}

    All cash flows (costs and income) in \euro:
    \begin{table}[!t]
    \begin{tabular}{lrrrrr}
      \toprule
      year $t$ &  0 & 1 & 2 & $\cdots$ & 20 \\
      \midrule
      Capital costs $I_t$ & 80.000  & 0 & 0  & & 0\\
      Operating costs $B_t$ & 0 & 2.000 & 2.000 & & 2.000  \\
      Income $U_t$ & 0 & 10.000  & 10.000 & & 10.000 \\
      \midrule
      Net cash flow $U_t - I_t - B_t$ & -80.000 & 8.000 & 8.000 & & 8.000 \\
      \midrule
      Discount factor $\frac{1}{(1+r)^t}$ & 1 & $\frac{1}{(1+r)}$ & $\frac{1}{(1+r)^2}$ & &  $\frac{1}{(1+r)^{20}}$ \\
      \bottomrule
    \end{tabular}
    \end{table}


\end{frame}



\begin{frame}
  \frametitle{NPV simplification}

  If investments only occur in the first year, and the costs and income for the following years are constant, we can simplify the NPV formula:
  \begin{equation*}
    NPV = -I_0 + (U - V - B) \sum_{t=1}^T \frac{1}{(1+r)^t}
  \end{equation*}
  The sum $\sum$ is called the \alert{Present Value Factor} $PVF(r,T)$.

  For a geometric series with $|q| < 1$ we have $\sum_{n=0}^\infty q^n = \frac{1}{1-q}$.
  For $q= (1+r)^{-1}$ we can simplify the formula
  \begin{align*}
    PVF(r,T) & =  \sum_{t=1}^T \frac{1}{(1+r)^t} \\
             & = \left[\frac{1}{(1+r)} - \frac{1}{(1+r)^{T+1}}  \right] \sum_{t=0}^\infty \frac{1}{(1+r)^t}
              = \left[\frac{1}{(1+r)} - \frac{1}{(1+r)^{T+1}}  \right] \frac{1}{1 - (1+r)^{-1}} \\
             & = \left[\frac{1}{(1+r)} - \frac{1}{(1+r)^{T+1}}  \right] \frac{1+r}{1 + r  - 1}  = \frac{1}{r} \left[1  - \frac{1}{(1+r)^{T}}  \right]
  \end{align*}

\end{frame}


\begin{frame}
  \frametitle{Example: Rooftop photovoltaic unit}

  For our example with $r= 0.05$
  \begin{align*}
    NPV & = -80.000 + (10.000 - 2.000)  \cdot \frac{1}{r} \left[1  - \frac{1}{(1+r)^{T}}  \right] \\
    & = -80.000 + 8.000*12.5 \\
    & = 19698
  \end{align*}

  \alert{Conclusion:} It's worthwile to invest in the photovoltaic unit!

  \pause

  NB: The calculation is very sensitive to the interest rate, e.g. with $r = 0.08$
  \begin{align*}
    NPV & = -80.000 + 8.000*9.8 \\
    & = -1.454
  \end{align*}

  \alert{Conclusion:} The investment is not worthwhile.

\end{frame}

\begin{frame}
  \frametitle{Return On Investment (ROI)}

  The expected return or \alert{Return On Investment (ROI)} is the required interest rate to reach the point $NPV = 0$.

  In our example you can either experiment or use the Newton-Raphson algorithm to determine the ROI $r$
  \begin{equation*}
    0 = NPV = -I_0 + (U - V - B) \sum_{t=1}^T \frac{1}{(1+r)^t}
  \end{equation*}

  In our example we find an ROI of $r=7.75$\%.

\end{frame}



\begin{frame}
  \frametitle{German example figures for electricity production technologies in 2018}

  WACC is the \alert{Weighted Average Cost of Capital} over the bank interest rate for borrowed capital (Fremdkapital) and the investor's ROI on their own investment (Eigenkapital).

  \centering
  \includegraphics[width=14cm]{stromgestehungskosten.png}


  \source{\href{https://www.ise.fraunhofer.de/content/dam/ise/de/documents/publications/studies/DE2018_ISE_Studie_Stromgestehungskosten_Erneuerbare_Energien.pdf}{Fraunhofer ISE Stromgestehungskosten 2018}}

\end{frame}


\begin{frame}
  \frametitle{Warning: Discounting over long time periods}

  Over long time periods the discounting can have a very large effect....

  \begin{columns}[T]
\begin{column}{9cm}
  \centering
  \includegraphics[width=9.5cm]{discounting.pdf}
\end{column}
\begin{column}{5cm}
  \begin{itemize}
  \item Long-term benefits aren't seen, e.g. long production life of nuclear power plants or benefits of long-lived efficiency measures
  \item Long-term costs are also suppressed, e.g. decommissioning, waste disposal, climate damages
  \item This is a \alert{controversial topic!}
  \end{itemize}
\end{column}
  \end{columns}

\end{frame}



\begin{frame}
  \frametitle{Programming example: photovoltaic plant}

  \centering
  \includegraphics[width=10.5cm]{PV_programming_example.png}


\end{frame}

\begin{frame}
  \frametitle{Programming example: nuclear plant}

  \centering
  \includegraphics[width=10.5cm]{nuclear_programming_example.png}


\end{frame}




\begin{frame}
  \frametitle{Summary}

  \begin{itemize}
  \item Future income or costs are worth less from today's point of view
  \item To calculate the \alert{present value}, multiply the cash flow in year $t$ by the \alert{discount factor}  $\frac{1}{(1+r)^t}$
  \item To calculate the \alert{net present value (NPV)} for an investment, sum the present values of all income and costs
  \item If  $NPV > 0$, the investment is worthwhile compared to investing with interest rate $r$
  \item For two different investments, a higher NPV should be preferred
  \item \alert{Long-term} costs or benefits are \alert{suppressed} by discounting
  \end{itemize}

\end{frame}


\section{Levelised Cost Of Electricity (LCOE)}




\begin{frame}
  \frametitle{Levelised Cost Of Energy (LCOE)}

  You can also solve for the market value or feed-in tariff that's
  necessary to cover all the costs of the investment, i.e. the point
  where the present value of all income balances the present value of
  all costs.  You solve for the price $\lambda$ such that
  \begin{equation*}
   0 = NPV = -I_0 + (\lambda Q - oQ - B) PVF(r,T)
  \end{equation*}
  (using $V = oQ$). We find:
  \begin{equation*}
    \lambda = \frac{1}{Q} \left( \frac{I_0}{PVF(r,T)} + B + oQ \right) = \frac{1}{Q} \left( \frac{I_0}{PVF(r,T)} + B  \right) + o
  \end{equation*}

  In our example we find a price of $\lambda =$ 89 \euro/MWh for $i=0.05$. % One could reduce the feed-in tariff from 100 \euro/MWh to this value!

  This value corresponds to the average long-term costs of the unit, since we've divided the total yearly costs by the total production $Q$. It is called the
the \alert{Levelised Cost Of Energy (LCOE)}.

  It is also called the  \alert{Long-Run Marginal Cost (LMRC)}, since we've added to the short-run marginal cost $o$ an annualised contribution to the capital cost and the operating costs.

  Check: The higher $I_0$ or $B$ are, the higher the LCOE. The higher $Q$ is, the lower the LCOE.
\end{frame}



\begin{frame}
  \frametitle{Annuity}

  The \alert{annuity} is the annualised investment cost $a = \frac{I_0}{PVF(r,T)}$ and
  $a(r,T) = \frac{1}{PVF(r,T)}$ is the \alert{annuity factor}, which spreads the capital costs $I_0$ evenly over the operational years of the investment (like a mortgage for a house).

  For a loan $I_0$ from the bank, the bank is compensated for the \alert{opportunity cost} of investing elsewhere at a rate of $r$ by an annual fixed sum $a$ so that the NPV for the bank is zero
  \begin{equation*}
    0 = NPV  = -I_0 + \sum_{t=0}^T \frac{a}{(1+r)^t} = -I_0 + PVF(r,T) \frac{I_0}{PVF(r,T)}
  \end{equation*}

  The formula for the annuity factor is derived from that for the PVF:
  \begin{equation*}
     a(r,T)  = \frac{1}{PVF(r,T)} = \frac{ r}{ 1 - (1+r)^{-T}}
  \end{equation*}


\end{frame}



\begin{frame}[fragile]
  \frametitle{Examples of annuity factor}

  \begin{columns}[T]
\begin{column}{6.5cm}

  AF = Annuity Factor, $a(r,T)$
  \begin{table}[!t]
	\centering
	\begin{tabular}{@{}rrr@{}}
\toprule
Lifetime $T$  &  Discount Rate $r$ & AF $a(r,T)$ \\
years & \% & per unit \\
\midrule
20 & 0 & 0.05 \\
20 & 5 & 0.08 \\
20 & 10 & 0.12 \\
20 & 20 & 0.21 \\
40 & 0 & 0.025 \\
40 & 5 & 0.06 \\
40 & 10 & 0.10 \\
40 & 20 & 0.20 \\
\bottomrule
	\end{tabular}
\end{table}
\end{column}
\begin{column}{7.5cm}
  Things to notice:
  \begin{itemize}
  \item AF reduce to $1/T$ in limit $r \to 0$
  \item AF climbs steeply with $r$
  \item For long lifetimes, AF is similar to short lifetimes for high $r$ - in reality investors try to pay off investments faster than lifetime
  \item In reality, an investor would provide some capital themselves, e.g. 10-20\% of the capital cost, and borrow the rest from the bank. The weighted average of the investor's desired internal rate of return and that of the bank loan is the \alert{weighted average cost of capital} (WACC).
  \end{itemize}
\end{column}
  \end{columns}
\end{frame}




\begin{frame}[fragile]
  \frametitle{Parameters for different generation technologies}

  Here are some typical investment and operational parameters projected for 2020:
  \vspace{-.5cm}
  \begin{table}[!t]
	\centering
	\begin{tabular}{@{}lrrrrrrr@{}}
\toprule
Source   & Lifetime & Capital Cost &  Fix O\&{}M &  Var O\&{}M & $\eta$ & Fuel Cost & Marg. Cost \\
& years & \euro kW$^{-1}$  & \euro kW$^{-1}$a$^{-1}$ &  \euro MWh$_{\textrm{el}}^{-1}$ & [\%] & \euro/MWh$_{\textrm{th}}$ & \euro/MWh$_{\textrm{el}}$ \\
\midrule
Hard Coal    & 40 & 1200 & 30 & 6 & 39 & 10 & 32 \\
Gas OCGT     & 30 & 400  & 15 & 3 & 39 & 20 & 54 \\
Gas CCGT     & 30 & 800  & 20 & 4 & 60 & 20 & 37 \\
Nuclear      & 40-60 & 6000  & 0 & 6 & 33 & 3.3 & 16 \\
Wind Onshore & 25 & 1240 & 35 & 0 & & 0 & 0 \\
Solar PV     & 25 & 750  & 25 & 0 & & 0 & 0 \\
\bottomrule
	\end{tabular}
\end{table}
O\&{}M = Operation and Maintenance, Var. = Variable, Fix. = Fixed, $\eta$ = efficiency

For a plant with capacity $G_s$ in MW and yearly production $Q$ in MWh$_{\textrm{el}}$, we have $I_0 = 1000 \cdot G_s \cdot (\textrm{Capital Cost})$, $B = 1000\cdot G_s \cdot (\textrm{Fix O\&{}M})$, $V = Q\cdot o$ where o is the marginal cost $o =  (\textrm{Marg. Cost}) = (\textrm{Var O\&{}M}) + (\textrm{Fuel Cost})/\eta$.

\source{\href{https://www.diw.de/documents/publikationen/73/diw_01.c.424566.de/diw_datadoc_2013-068.pdf}{DIW Data Documentation, 2013}}
\end{frame}




\begin{frame}
  \frametitle{LCOE for dispatchable generators depends on capacity factor}

  The LCOE had the form (Marg. Cost) + (Yearly Fixed Costs)/(Yearly Production). Therefore it decreases with increasing capacity factor:

  \begin{columns}[T]
    \begin{column}{8.5cm}
      \vspace{.5cm}
  \centering
  \includegraphics[width=8cm]{lcoe_cf.pdf}
\end{column}
\begin{column}{6.5cm}
  \begin{itemize}
  \item LCOE $>$ marginal cost
  \item LCOE starts high then reduces as fixed costs are spread over more hours
  \item There are \alert{crossing points} where some types of generators become cheaper for a given capacity factor
  \item NB: All generators need downtime for regular maintenance, so cf $< 0.9$
  \item NB: Carbon pricing would alter this graphic by adding to the marginal cost
  \end{itemize}
\end{column}
  \end{columns}

\end{frame}


\begin{frame}{LCOE for wind and solar depends on location: worldwide auction results 2017}

  \centering
  \includegraphics[width=11.5cm]{baringa-auctions.png}

  \source{\href{https://www.baringa.com/getmedia/99d7aa0f-5333-47ef-b7a8-1ca3b3c10644/Baringa_Scottish-Renewables_UK-Pot-1-CfD-scenario_April-2017_Report_FINA/}{Baringa Partners LLP 2017}}

\end{frame}


\begin{frame}[fragile]
  \frametitle{Levelised Cost of Electricity Since 2009 in US}

  NB: Treat with care since LCOE doesn't take account of time or place of generation!

  \centering
  \includegraphics[width=14cm]{lazard-cropped.jpg}

  \source{\href{https://www.lazard.com/perspective/levelized-cost-of-energy-2017/}{Lazard's LCOE Analysis V11}}

\end{frame}



\section{Duration Curves and Capacity Factors: Examples from Germany in 2015}

\begin{frame}{Load curve}

  Here's the electrical demand (load) in Germany in 2015:

  \centering
  \includegraphics[width=11cm]{load-2015}

  \raggedright

  To understand this curve better and its implications for the market,
  it's useful to stack the hours of the year from left to right in
  order of the amount of load.

\end{frame}


\begin{frame}{Load duration curve}

  This re-ordering is called a \alert{duration curve}.\\
  For the load it's the \alert{load duration curve}.

  \centering
  \includegraphics[width=13cm]{load-duration-2015}

\end{frame}

\begin{frame}{Nuclear curve}

  Can do the same for nuclear output:

  \centering
  \includegraphics[width=10.5cm]{Nuclear-2015}

\end{frame}

\begin{frame}{Nuclear duration curve}

  Duration curve is pretty flat, because it is economic to run nuclear almost all the time as  \alert{baseload plant}:

  \centering
  \includegraphics[width=13cm]{Nuclear-duration-2015}

  \raggedright
  The equivalent fraction of time that the plants run at full capacity over the year is the \alert{capacity factor} - nuclear has a high capacity factor, usually around 70-90\%.
\end{frame}


\begin{frame}{Gas curve}

  Can do the same for gas output:

  \centering
  \includegraphics[width=13cm]{Gas-2015}

\end{frame}

\begin{frame}{Gas duration curve}

  Duration curve is partially flat (for heat-driven CHP) and partially peaked (for \alert{peaking plant}):

  \centering
  \includegraphics[width=12cm]{Gas-duration-2015}

  \raggedright
  The capacity factor for gas is much lower - more like 20\%.
\end{frame}

\begin{frame}{Price curve}

  Can do the same for price during the year:

  \centering
  \includegraphics[width=13cm]{price-2015}

\end{frame}

\begin{frame}{Price duration curve}

  By ordering we get the \alert{price duration curve}:

  \centering
  \includegraphics[width=13cm]{price-duration-2015}

\end{frame}



\begin{frame}{Question}



  Now we are in a position to consider the questions:

  \begin{itemize}
    \item What determines the distribution of investment in different generation technologies?
    \item How is it connected to variable costs, capital costs and capacity factors?
  \end{itemize}

  We will find the price and load duration curves very useful.
\end{frame}


\section{Investment Optimisation: Generation}

\begin{frame}[fragile]
  \frametitle{Investment optimisation}

 Now we also optimise \alert{investment} in the \alert{capacities} of generators,
  storage and network lines for the \alert{whole system} not just a single plant operator, to maximise \alert{long-run efficiency}.

  We will promote the capacities $G_{i,s}$, $G_{i,r,*}$, $E_{i,r}$ and
  $F_{\ell}$ to optimisation variables.

  For generation investment, we want to answer the following questions:
  \begin{itemize}
    \item What determines the distribution of investment in different generation technologies?
    \item How is it connected to variable costs, capital costs and capacity factors?
  \end{itemize}

  We will find price and load duration curves very useful.
\end{frame}


\begin{frame}{Definition of long-run efficiency}

  Up until now we have considered \alert{short-run} equilibria that
  ensure \alert{short-run} efficiency (static), i.e. they make the best use of
  presently available productive resources.

  \alert{Long-run} efficiency (dynamic) requires in addition the optimal
  investment in productive capacity.

  Concretely: given a set of options, costs and constraints for different
  generators (nuclear/gas/wind/solar) what is the optimal generation
  portfolio for maximising long-run welfare?

  From an indivdual generators' perspective: how best should I invest in extra capacity?

  We will show again that with perfect competition and no barriers to
  entry, the system-optimal situation can be reached by individuals
  following their own profit.

\end{frame}



\begin{frame}
  \frametitle{Baseload versus Peaking Plant}

  \alert{Load} (= Electrical Demand) is low during night; in Northern Europe in the winter, the peak is in the evening.

  To meet this load profile, cheap \alert{baseload} generation runs the whole time; more expensive \alert{peaking plant} covers the difference.

  \centering
  \includegraphics[width=9cm]{baseload-peaking}

  \source{Tom Brown}

\end{frame}




\begin{frame}{Different types of generators}

\ra{1.1}
\begin{table}[!t]
	\centering
	\begin{tabular}{@{}llllll@{}}
\toprule
Fuel/Prime  & Marginal & Capital & Controllable & Predictable & CO2 \\
mover & cost & cost & & days ahead & \\
\midrule
Oil & V. High & Low & Yes & Yes & Medium \\
Gas OCGT & High & Low & Yes & Yes & Medium \\
Gas CCGT & Medium & Medium & Yes & Yes & Medium \\
Hard Coal & Medium & Lowish & Yes & Yes & High \\
Brown Coal & Low & Medium & Yes & Yes & High \\
Nuclear & V. Low & High & Partly & Yes & Zero \\
Hydro dam & Zero & High & Yes & Yes & Zero \\
Wind/Solar & Zero & High & Down & No & Zero \\
\bottomrule
	\end{tabular}
\end{table}

\end{frame}


\begin{frame}{System-optimal generator capacities and dispatch}

  Suppose we have generators labelled by $s$ at a single node with \alert{marginal costs} $o_s$ from each unit of
  production $g_{s,t}$ and \alert{specific capital costs} $c_s$ from fixed costs
  regardless of the rate of production (e.g. investment in building
  capacity $G_s$).   For a variety of demand values $d_t$ that occur with probability $p_t$ ($\sum_t p_t = 1$)   we optimise the total \alert{average hourly system costs}
  \begin{equation*}
    \min_{\{g_{s,t}\},\{G_s\}}  \left[\sum_{s}c_s G_s +  \sum_{s,t} p_t o_{s} g_{s,t} \right]
  \end{equation*}
  such that (rescaling the KKT multipliers by $p_t$ to simplify later formulae)
  \begin{align*}
    \sum_s g_{s,t} & = d_t  \hspace{1cm}\leftrightarrow\hspace{1cm} p_t \l_t \\
    - g_{s,t}  & \leq  0  \hspace{1cm}\leftrightarrow\hspace{1cm} p_t \ubar{\m}_{s,t} \\
    g_{s,t} - G_s  & \leq 0  \hspace{1cm}\leftrightarrow\hspace{1cm} p_t \bar{\m}_{s,t}
  \end{align*}

  We will also allow load-shedding with a `dummy' generator $s=S$,
  $o_S = V$ (Value of Lost Load), $c_S=0$ (the capacity to shed load
  is assumed not to cost anything).

\end{frame}



\begin{frame}{Beware units and scaling}

  We've chosen the units here so that the total objective function has units \euro{}h$^{-1}$, the \alert{average hourly system costs}.

  $\sum_{s,t} p_t o_{s} g_{s,t}$ is the expectation value of the hourly production costs. $g_{s,t}$ has units MW, $o_{s}$ has units \euro{}(MWh)$^{-1}$.

  $c_sG_s$ is the investment cost averaged over each hour, i.e. the annuity divided by 8760, $\frac{a(r,T)I_0}{8760}$ (we can also add the fixed O\&{}M costs $B$ to it). $G_s$ has units MW, $c_s$ has units
  \euro{}MW$^{-1}$h$^{-1}$.

  We could have instead optimised \alert{average yearly system costs}, then $c_s G_s$ would simply be the annuity, and instead of weighting with $p_t$ such that $\sum_t p_t = 1$, we replace it with a weighting $w_t$ such that $\sum_t w_t = 8760$. In this case, the total objective would have units \euro{}a$^{-1}$.

\end{frame}

\begin{frame}{System-optimal generator capacities and dispatch}

  Stationarity gives us for each $s$ and $t$:
  \begin{align*}
        0 = \frac{\d \cL}{\d g_{s,t}}  = p_t \left(o_s - \l_t^* - \bar{\m}^*_{s,t} + \ubar{\m}^*_{s,t} \right)
  \end{align*}
  and for each $s$:
  \begin{align*}
        0 = \frac{\d \cL}{\d G_{s}}  = c_s + \sum_t p_t \bar{\m}^*_{s,t}
  \end{align*}
  and from complementarity we get
  \begin{align*}
    \bar{\m}^*_{s,t}(g_{s,t}^* - G_s^*) & = 0 \\
    \ubar{\m}^*_{s,t}g_{s,t}^* & = 0
  \end{align*}
  and dual feasibility (for minimisation) $ \bar{\m}^*_{s,t},  \ubar{\m}^*_{s,t} \leq 0$.
\end{frame}



\begin{frame}{System-optimal generator capacities and dispatch}

  The solution for the dispatch $g_{s,t}^*$ is exactly the same as
  without capacity optimisation. For each $t$, find the generator $m$
  where the supply curve intersects with the demand $d_t$, i.e. the $m$
  where   $\sum_{s=1}^{m-1} G_s < d_t < \sum_{s=1}^{m} G_s$.

  For $s < m$ we have $g_{s,t}^* = G_s^*$, $\ubar{\m}^*_{s,t} = 0$,
  $\bar{\m}^*_{s,t} = o_s - \l_t^* \leq 0$.

  For $s = m$ we have $g_{m,t}^* = d_t - \sum_{s=1}^{m-1} G_s^*$ to cover
  what's left of the demand. Since $0 < g_{m,t}^* < G_m$ we have
  $\ubar{\m}^*_{m,t} = \bar{\m}^*_{m,t} = 0$ and therefore $\l_t^* = o_m$.

  For $s > m$ we have $g_{s,t}^* = 0$,
  $\ubar{\m}^*_{s,t} = \l_t^*-o_s \leq 0 $, $\bar{\m}^*_{s,t} = 0$.

  What about the $G_s^*$?


\end{frame}



\begin{frame}{System-optimal generator capacities and dispatch}

  The $G_s^*$ are determined implicitly based on the interactions between costs and prices.

  From stationarity we had the relation
    \begin{align*}
      c_s  = - \sum_t p_t \bar{\m}^*_{s,t}
    \end{align*}
    The $\bar{\m}^*_{s,t}$ were only non-zero with $\l^*_{t} > o_s$ so we can re-write this as
    \begin{align*}
      c_s  = \sum_{t| \l^*_{t} > o_s} p_t (\l^*_{t} - o_s)
    \end{align*}

  `Increase capacity until marginal increase in profit equals the cost of extra capacity.'

\end{frame}


\begin{frame}{Multiple price duration}

  The optimal mix of generation is where, for each generation type,
  the area under the price–duration curve and above the variable cost
  of that generation type is equal to the fixed cost of adding
  capacity of that generation type.

  \centering
    \begin{tikzpicture}
\node[anchor=south west,inner sep=0] (image) at (0,0) {\includegraphics[width=8.5cm]{multiple-price-duration-clean}};
\draw (0.6,1.8) node{$o_1$};
\draw (0.6,2.6) node{$o_2$};
\draw (0.6,3.4) node{$o_3$};
  \end{tikzpicture}

  \source{Biggar and Hesamzadeh, 2014}
\end{frame}




\begin{frame}{Multiple generators with inelastic demand}

  Assume again we have $o_1 \leq o_2 \leq \cdots \leq o_S = V$ and $K_p = \sum_{s=1}^p G_s$
  then:
  \begin{equation*}
    \l_t = \left\{ \begin{array}{ll}
      V& \textrm{ for } d_t > K_{S-1} \\
      o_s & \textrm{ if } K_{s-1} < d_t \leq K_s, \hspace{1cm} \textrm{ for } s = 1, \dots S-1
    \end{array}\right.
  \end{equation*}

  Looking at the area under the price duration curve but above the
  variable cost, we then find:
  \begin{equation*}
    c_s = (V-o_s)P(d>K_{S-1}) + \sum_{j=s+1}^{S-1} (o_j-o_s) P(K_{j-1} < d \leq K_j)
  \end{equation*}


\end{frame}


\begin{frame}{Screening curve}

  These equations can be rewritten recursively using the substitution
  $\theta_s = P(d > K_s)$:
  \begin{align*}
    c_{S-1} + \theta_{S-1} o_{S-1} &= V\theta_{S-1} \\
    c_s + \theta_s o_s &= c_{s+1} + \theta_s o_{s+1} \hspace{1cm} \forall s = 1, \dots S-2
  \end{align*}
  The first equation can be solved to find $\theta_{S-1}$, then the other equations can be solved recursively to find the remaining $\theta_s$. The $\theta_s$ correspond to the optimal \alert{capacity factors} of each type of generator, which correspond to the fraction  of time the generator runs at full power.


\end{frame}


\begin{frame}{Screening curve}

  The costs as a function of the capacity factors can be drawn
  together as a \alert{screening curve} (more expensive options are
  \emph{screened} from the optimal inner polygon).

  The intersection points determine the optimal capacity factors and hence, using the load duration curve, the optimal capacities of each generator type.

  \centering
  \begin{tikzpicture}
\node[anchor=south west,inner sep=0] (image) at (0,0) {\includegraphics[width=10.5cm]{screening-curve-clean}};
\draw (0.2,1) node{$c_2$};
\draw (0.2,3) node{$c_1$};
  \end{tikzpicture}

  \source{Biggar and Hesamzadeh, 2014}
\end{frame}



\begin{frame}{Screening curve versus Load duration}


  \centering
  \begin{tikzpicture}
\node[anchor=south west,inner sep=0] (image) at (0,0) {\includegraphics[width=6.5cm]{screening-load-duration-clean}};
\draw (1.9,4.2) node{$c_2$};
\draw (1.9,5.6) node{$c_1$};
  \end{tikzpicture}

  \source{Biggar and Hesamzadeh, 2014}
\end{frame}


\begin{frame}{Example: 2 generation technologies and load shedding}


Suppose that electrical demand is inelastic with a demand-duration curve given by $d(x)=1000-1000x$ for $0\leq x \leq 1$. Suppose that there are two different types of generation with variable costs of 2 and 12~\euro/MWh respectively, together with load-shedding at a cost of 1012~\euro/MWh. The fixed costs of the two generation types are 15 and 10~\euro/MWh respectively. See the below table for a summary of the costs.



  \ra{1.1}
  \begin{table}[!h]
    \begin{tabular}{lrr}
      \toprule
      Generator & $o_s$ [\euro/MWh] &  $c_s$ [\euro/MW/h] \\
      \midrule
      A & 2 & 15 \\
      B & 12 & 10 \\
      LS & 1012 & 0 \\
      \bottomrule
    \end{tabular}
  \end{table}


\end{frame}

\begin{frame}{Example: 2 generation technologies and load shedding}
\begin{enumerate}
  \item What is the interpretation of the demand-duration curve?
  \item Below which capacity factor $x_1$ is it cheaper to run Generator B rather than to run Generator A?
  \item Below which capacity factor $x_0$ is it cheaper to shed load than to run Generator B?
  \item Plot the costs as a function of
    $x$ and mark these intersection points on a screening curve.
  \item Find the optimal capacities of Generators A and B in this market.
\end{enumerate}
\end{frame}



\begin{frame}{Example: 2 generation technologies and load shedding}

  For the solution see the flipchart photos at \url{https://nworbmot.org/courses/esm-2018/board/}.

  To find $x_1$, solve for the intersection of Generator A's cost curve with Generator B's cost curve as a function of capacity factor:
  \begin{equation*}
    c_A + x_1 o_A = c_B + x_1 o_B
  \end{equation*}
  This gives $x_1 = 0.5$. At this point the demand is $d(0.5) = 500$~MW therefore
  \begin{equation*}
    G_A = 500\textrm{ MW}
  \end{equation*}

  To find $x_0$, solver for where Generator B crosses the load-shedding line:
  \begin{equation*}
    c_B + x_0 o_B = c_{LS} + x_0 o_{LS}
  \end{equation*}
  This gives $x_0 = 0.01$. At this point the demand is $d(0.5) = 990$~MW so:
  \begin{equation*}
    G_A + G_B = 990\textrm{ MW}
  \end{equation*}
  i.e. $G_B = 490$ MW and $G_{LS} = 10$ MW.

\end{frame}

\section{Investment Optimisation: Transmission}


\begin{frame}[fragile]
  \frametitle{Investment optimisation: transmission}

  As before, our approach to the question of \alert{``What is the
    optimal amount of transmission''} is determined by the most
  efficient long-term solution, i.e. the infrastructure investement
  that maximising social welfare over the long-run.


   Promote $F_\ell$ to an optimisation variable with capital cost
   $c_\ell$.


  In brief: Exactly as with generation dispatch and investment, we
  continue to invest in transmission until the marginal benefit of
  extra transmission (i.e. extra congestion rent for extra capacity)
  is equal to the marginal cost of extra transmission. This determines
  the optimal investment level.

  For the generator case we had $c_s = - \sum_t p_t \bar{\mu}_{s,t}^*$ where $\bar{\mu}_{s,t}^*$ were the shadow prices of the constraints  $g_{s,t} \leq G_s$.

  For the transmission line we have $f_{\ell} \leq F_{\ell}$ ($\bar{\mu}_{\ell,t}^*$) and $-f_{\ell} \leq F_{\ell}$ ($\ubar{\mu}_{\ell,t}^*$) so we get
  \begin{equation*}
    c_\ell = - \sum_t p_t \left(\bar{\mu}_{\ell,t}^* + \ubar{\mu}_{\ell,t}^* \right)
  \end{equation*}

\end{frame}

\end{document}
