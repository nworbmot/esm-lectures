% NB: use pdflatex to compile NOT pdftex.  Also make sure youngtab is
% there...

% converting eps graphics to pdf with ps2pdf generates way too much
% whitespace in the resulting pdf, so crop with pdfcrop
% cf. http://www.cora.nwra.com/~stockwel/rgspages/pdftips/pdftips.shtml




\documentclass[10pt,aspectratio=169,dvipsnames]{beamer}
\usetheme[color/block=transparent]{metropolis}

\usepackage[absolute,overlay]{textpos}
\usepackage{booktabs}
\usepackage[utf8]{inputenc}

\usepackage{tikz}


\usepackage[scale=2]{ccicons}

\usepackage[official]{eurosym}


%use this to add space between rows
\newcommand{\ra}[1]{\renewcommand{\arraystretch}{#1}}


\newcommand{\R}{\mathbb{R}}



\setbeamerfont{alerted text}{series=\bfseries}
\setbeamercolor{alerted text}{fg=Mahogany}
\setbeamercolor{background canvas}{bg=white}



\def\l{\lambda}
\def\m{\mu}
\def\d{\partial}
\def\cL{\mathcal{L}}
\def\co2{CO${}_2$}
\def\bra#1{\left\langle #1\right|}
\def\ket#1{\left| #1\right\rangle}
\newcommand{\braket}[2]{\langle #1 | #2 \rangle}
\newcommand{\norm}[1]{\left\| #1 \right\|}
\def\corr#1{\Big\langle #1 \Big\rangle}
\def\corrs#1{\langle #1 \rangle}


\def\mw{\text{ MW}}
\def\mwh{\text{ MWh}}
\def\emwh{\text{ \euro/MWh}}

\newcommand{\ubar}[1]{\text{\b{$#1$}}}


% for sources http://tex.stackexchange.com/questions/48473/best-way-to-give-sources-of-images-used-in-a-beamer-presentation

\setbeamercolor{framesource}{fg=gray}
\setbeamerfont{framesource}{size=\tiny}



\newcommand{\source}[1]{\begin{textblock*}{5cm}(10.5cm,8.35cm)
    \begin{beamercolorbox}[ht=0.5cm,right]{framesource}
        \usebeamerfont{framesource}\usebeamercolor[fg]{framesource} Source: {#1}
    \end{beamercolorbox}
\end{textblock*}}

\usepackage{hyperref}

\usepackage{tikz}


\usepackage[europeanresistors,americaninductors]{circuitikz}


%\usepackage[pdftex]{graphicx}


\graphicspath{{graphics/}}

\DeclareGraphicsExtensions{.pdf,.jpeg,.png,.jpg,.gif}



\def\goat#1{{\scriptsize\color{green}{[#1]}}}



\let\olditem\item
\renewcommand{\item}{%
\olditem\vspace{5pt}}


\title{Energy System Modelling\\ Summer Semester 2020, Lecture 7}
%\subtitle{---}
\author{
  {\bf Dr. Tom Brown}, \href{mailto:tom.brown@kit.edu}{tom.brown@kit.edu}, \url{https://nworbmot.org/}\\
  \emph{Karlsruhe Institute of Technology (KIT), Institute for Automation and Applied Informatics (IAI)}
}

\date{}


\titlegraphic{
  \vspace{0cm}
  \hspace{10cm}
    \includegraphics[trim=0 0cm 0 0cm,height=1.8cm,clip=true]{kit.png}

\vspace{5.1cm}

  {\footnotesize

  Unless otherwise stated, graphics and text are Copyright \copyright Tom Brown, 2020.
  Graphics and text for which no other attribution are given are licensed under a
  \href{https://creativecommons.org/licenses/by/4.0/}{Creative Commons
  Attribution 4.0 International Licence}. \ccby}
}

\begin{document}

\maketitle


\begin{frame}

  \frametitle{Table of Contents}
  \setbeamertemplate{section in toc}[sections numbered]
  \tableofcontents[hideallsubsections]
\end{frame}


\section{Introduction to Electricity Markets}


\begin{frame}
  \frametitle{The Economic Operation of the Electricity Sector}


  Given the many different ways of consuming and generating
  electricity:

  \begin{itemize}
    \item   What is the \alert{most efficient} way to deploy consuming and generating
  assets in the short-run?
    \item How should we invest in assets in the long-run to \alert{maximise economic welfare}?
  \end{itemize}

  The operation of electricity markets is intimately related to
  \alert{optimisation}.

  In the past and still in many countries today, electricity was
  provided centrally by `vertically-integrated' monopoly utilities
  that owned generating assets, the electricity networks and
  retailing.  Given that these utilities owned all the infrastructure,
  it was hard for third-party generators to compete, even if they were
  allowed to.

  From the 1980s onwards, countries began to liberalise their
  electricity sectors, separating generation from transmission, and
  allowing regulated competition for generation in \alert{electricity markets}.
\end{frame}


\begin{frame}
  \frametitle{Electricity Markets}

  Electricity markets have several important differences compared to
  other commodity markets.

  At every instant in time, consumption must be balanced with
  generation.

  If you throw a switch to turn on a light, somewhere a generator will
  be increasing its output to compensate.

  If the power is not balanced in the grid, the power supply will
  collapse and there will be blackouts.

  It is not possible to run an electricity market for every single
  second, for practical reasons (the network must be checked for
  stability, etc.).

  So electricity is traded in blocks of time, e.g. hourly,
  14:00-15:00, or quarter-hourly, 14:00-14:15, well in advance of the
  time when it is actually consumed (based on forecasts).

  Further markets trade in backup balancing power, which step in if the
  forecasts are wrong.


\end{frame}



\begin{frame}
  \frametitle{Baseload versus Peaking Plant}

  \alert{Load} (= Electrical Demand) is low during night; in Northern Europe in the winter, the peak is in the evening.

  To meet this load profile, cheap \alert{baseload} generation runs the whole time; more expensive \alert{peaking plant} covers the difference.

  \centering
  \includegraphics[width=9cm]{baseload-peaking}

  \source{Tom Brown}

\end{frame}


\begin{frame}
  \frametitle{Effect of varying demand for fixed generation}

  \centering
  \includegraphics[width=9cm]{demand-supply}

  \source{Tom Brown}

\end{frame}




\begin{frame}
  \frametitle{Example market 1/3}

  \centering

  \includegraphics[width=9cm]{germany_generation-price-2015-01-06-2015-01-12}

\end{frame}

\begin{frame}
  \frametitle{Example market 2/3}

  \centering
  \includegraphics[width=9cm]{germany_generation-price-2015-03-15-2015-03-31}

\end{frame}

\begin{frame}
  \frametitle{Example market 3/3}

  \centering
  \includegraphics[width=9cm]{germany_generation-price-2015-04-07-2015-04-14}

\end{frame}


\begin{frame}
  \frametitle{Effect of varying renewables: fixed demand, no wind}

  \centering
  \includegraphics[width=9cm]{demand-supply-no_wind}

\end{frame}



\begin{frame}
  \frametitle{Effect of varying renewables: fixed demand, 35~GW wind}

  \includegraphics[width=9cm]{demand-supply-with_wind}

\end{frame}


\begin{frame}
  \frametitle{Spot market price development}

  As a result of so much zero-marginal-cost renewable feed-in, spot
  market prices steadily decreased until 2016. Since then prices have
  been rising due to rising gas and CO$_2$ prices.

  \includegraphics[width=11cm]{price_development-2018.png}

  \source{\href{https://www.agora-energiewende.de/fileadmin2/Projekte/2018/Jahresauswertung_2018/Agora-Annual-Review-2018_Energy-Transition-EN.pdf}{Agora Energiewende}}
\end{frame}



\section{Optimisation Revision}

\begin{frame}
  \frametitle{Optimisation problem}


We have an \alert{objective function} $f: \R^k \to \R$
\begin{equation*}
  \max_{x} f(x)
\end{equation*}
[$x = (x_1, \dots x_k)$] subject to some \alert{constraints} within $\R^k$:
\begin{align*}
  g_i(x) & = c_i \hspace{1cm}\leftrightarrow\hspace{1cm} \l_i \hspace{1cm} i = 1,\dots n \\
  h_j(x) & \leq d_j \hspace{1cm}\leftrightarrow\hspace{1cm} \m_j \hspace{1cm} j = 1,\dots m
\end{align*}

$\l_i$ and $\m_j$ are the \alert{KKT multipliers} (basically Lagrange multipliers) we introduce for
each constraint equation; it measures the change in the objective value of the optimal solution obtained by relaxing the constraint (shadow price).

\end{frame}



\begin{frame}
  \frametitle{KKT conditions}

The \alert{Karush-Kuhn-Tucker (KKT) conditions} are necessary conditions that an optimal solution $x^*,\m^*,\l^*$ always satisfies (up to some regularity conditions):
\begin{enumerate}
\item \alert{Stationarity}: For $l = 1,\dots k$
  \begin{equation*}
  \frac{\d \cL}{\d x_l} =   \frac{\d f}{\d x_l} - \sum_i \l_i^* \frac{\d g_i}{\d x_l}  - \sum_j \m_j^* \frac{\d h_j}{\d x_l} = 0
  \end{equation*}
    \item \alert{Primal feasibility}:
      \begin{align*}
        g_i(x^*) & = c_i \\
        h_j(x^*) &\leq d_j
      \end{align*}
    \item \alert{Dual feasibility}: $\m_j^* \geq 0$
    \item \alert{Complementary slackness}: $\m_j^* (h_j(x^*) - d_j) = 0$
\end{enumerate}

\end{frame}

\section{Electricity Markets from Perspective of Single Generators and Consumers}

\begin{frame}{Efficient Markets for the short-run}


  Assume investments already made in generators and and consumption assets (factories, machines, etc.).

  Assume all actors are price takers (i.e. nobody can exercise market power) and we have perfect competition.

  How do we allocate production and consumption in the most efficient way?


  I.e. we are interested in the \alert{short-run ``static'' efficiency}.

  \vspace{.7cm}

  (In contrast to \alert{long-run ``dynamic'' efficiency} where we also consider optimal investment in assets.)
\end{frame}





\begin{frame}
  \frametitle{Single Generator: Cost Function}

  Consider now the market from the point of view of a single generator.

  A generator has a \alert{cost or supply function} $C(g)$ in \euro/h,
  which gives the total operating costs (of fuel, etc.) for a given
  rate of electricity generation $g$~MW.

  Typically the generator has a higher cost for a higher rate of generation $g$,
  i.e. the first derivative is positive $C'(g) > 0$. For most generators the rate at which cost increases with rate of production itself increases as the rate of production increases, i.e. $C''(g) > 0$.

\end{frame}


\begin{frame}{Cost Function: Example}


  A gas generator has a cost function which depends on the
  rate of electricity generation $g$ [\euro/h] according to
  \begin{equation*}
    C(g) = 0.005~g^2  + 9.3~g + 120
  \end{equation*}


  \centering
  \includegraphics[width=6.3cm]{generator_cost_function}

  \raggedright
  Note that the slope is always positive and becomes more positive
  for increasing $g$. The curve does not start at the origin because of startup costs, no load costs, etc.
\end{frame}



\begin{frame}{Optimal generator behaviour}

  We assume that the generator is a \alert{price-taker},
  i.e. they cannot influence the price by changing the amount they
  generate.

  Suppose the market price is $\lambda$ \euro/MWh. For a generation
  rate $g$, the revenue is $\l g$ and the generator should adjust their
generation rate $g$ to maximise their \alert{net generation surplus}, i.e. their profit:
  \begin{equation*}
    \max_g \left[ \l g - C(g) \right]
  \end{equation*}

  This optimisation problem is optimised for $g=g^*$ where
  \begin{equation*}
    C'(g^*) \equiv \frac{dC}{dg} (g^*) =  \l
  \end{equation*}

  [Check units: $\frac{dC}{dg}$ has units $\frac{\textrm{\euro/h}}{\textrm{MW}} = $ \euro/MWh.]

  I.e. the generator increases their output until they make a net loss for any increase of generation.

  $C'(g)$ is known as the \alert{marginal cost curve}, which shows,
  for each rate of generation $g$ what price $\l$ the generator should be
  willing to supply at.


\end{frame}



\begin{frame}{Marginal cost function: Example}

  For our example the marginal function is given by
  \begin{equation*}
    C'(g) = 0.01~g + 9.3
  \end{equation*}

  \centering
  \includegraphics[width=7cm]{marginal_cost_function}

\end{frame}





\begin{frame}{Generator surplus}

  The area under the curve is generator costs, which as the integral
  of a derivative, just gives the cost function $C(g)$ again, up to a
  constant.


  The \alert{generator surplus} is the profit the generator makes by
  having costs below the electricity price.

  \centering
  \includegraphics[width=7cm]{generator_surplus}

\end{frame}



\begin{frame}{Limits to generation}


  Note that it is quite common for generators to be limited by
  e.g. their capacity $G$, which may become a \alert{binding}, i.e. limiting factor before the price plays a role, e.g.
  \begin{equation*}
    g \leq G  \leftrightarrow \m
  \end{equation*}
  In the following case the optimal generation is at
  $g^* = G = $ 250~MW.   We have a \alert{binding} constraint and can define a \alert{shadow price} $\m$, which indicates the benefit of relaxing the constraint $\m^* = \l - C'(g^*)$.


  \centering
  \includegraphics[width=6.5cm]{generator_surplus-limited}

\end{frame}




\begin{frame}{Consumer behaviour: Theory}

  Suppose for some given period a consumer consumes electricity at a rate of
  $d$~MW.


  Their \alert{utility or value function} $U(d)$ in \euro/h is a
  measure of their benefit for a given consumption rate $d$.

  For a firm this could be the profit related to this electricity
  consumption from manufacturing goods.


  Typical the consumer has a higher utility for higher $d$, i.e. the
  first derivative is positive $U'(d) > 0$. By assumption, the rate of
  value increase with consumption decreases the higher the rate of
  consumption, i.e. $U''(d) < 0$.

\end{frame}


\begin{frame}{Utility: Example}


  A widget manufacturer has a utility function which depends on the
  rate of electricity consumption $d$ [\euro/h] as
  \begin{equation*}
    U(d) = 0.0667~d^3  - 8~d^2 + 300~d
  \end{equation*}


  \centering
  \includegraphics[width=6cm]{utility}

  \raggedright
  Note that the slope is always positive, but becomes less positive
  for increasing $d$.
\end{frame}


\begin{frame}{Optimal consumer behaviour}



  We assume that the consumer is a \alert{price-taker},
  i.e. they cannot influence the price by changing the amount they
  consume.

  Suppose the market price is $\lambda$ \euro/MWh. The consumer should
  adjust their consumption rate $d$ to maximise their \alert{net surplus}
  \begin{equation*}
    \max_d \left[U(d) - \l d \right]
  \end{equation*}

  This optimisation problem is optimised for $d=d^*$ where
  \begin{equation*}
    U'(d^*) \equiv \frac{dU}{dd} (d^*) =  \l
  \end{equation*}

  [Check units: $\frac{dU}{dd}$ has units $\frac{\textrm{\euro/h}}{\textrm{MW}} = $ \euro/MWh.]

  I.e. the consumer increases their consumption until they make a net loss for any increase of consumption.

  $U'(d)$ is known as the \alert{inverse demand curve} or \alert{marginal utility curve}, which shows,
  for each rate of consumption $d$ what price $\l$ the consumer should be
  willing to pay.


\end{frame}



\begin{frame}{Inverse demand function: Example}
  For our example the inverse demand function is given by
  \begin{equation*}
    U'(d) = 0.2~d^2 - 16~d + 300
  \end{equation*}

  \centering
  \includegraphics[width=6cm]{inverse_demand_function}

  \raggedright
  It's called the \emph{inverse} demand function, because the demand function is the function you get from reversing the axes.
\end{frame}


\begin{frame}{Inverse demand function: Example}


  The \alert{demand function} $D(\l)$ gives the demand $d$ as a function of the price $\l$. $D(U'(d)) = d$.

  For our example the demand function is given by
  \begin{equation*}
    D(\l) =  -((\l+20 )/0.2)^{0.5} +40
  \end{equation*}

  \centering
  \includegraphics[width=6cm]{demand_function}

\end{frame}



\begin{frame}{Gross consumer surplus}

  The area under the inverse demand curve is the \alert{gross consumer surplus},
  which as the integral of a derivative, just gives the utility
  function $U(d)$ again, up to a constant.

  \centering
  \includegraphics[width=6cm]{gross_consumer_surplus}

\end{frame}



\begin{frame}{Net consumer surplus}

  The more relevant \alert{net consumer surplus}, or just
  \alert{consumer surplus} is the net gain the consumer makes by
  having utility above the electricity price.

  \centering
  \includegraphics[width=6cm]{net_consumer_surplus}

\end{frame}


\begin{frame}{Limits to consumption}


  Note that it is quite common for consumption to be limited by other
  factors before the electricity price becomes too expensive, e.g. due
  to the size of electrical machinery. This gives an upper bound
  \begin{equation*}
    d \leq D  \leftrightarrow \m
  \end{equation*}
  In the following case the optimal consumption is at
  $d^* = D =$  10~MW. We have a \alert{binding} constraint and can define a \alert{shadow price} $\m$, which indicates the benefit of relaxing the constraint $\m^* = U'(d^*) - \l$.

  \centering
  \includegraphics[width=6cm]{net_consumer_surplus-limited}

\end{frame}



\begin{frame}
  \frametitle{Consumers can delay their consumption}

  Besides changing the amount of electricity consumption, consumers
  can also shift their consumption in time.

  For example electric storage heaters use cheap electricity at night
  to generate heat and then store it for daytime.

  The LHC particle accelerator does not run in the winter,
  when prices are higher (see
  \url{http://home.cern/about/engineering/powering-cern}).  Summer
  demand: 200~MW, corresponds to a third of Geneva, equal to peak
  demand of Rwanda (!); winter only 80~MW.


  \centering
  \includegraphics[width=6cm]{csm_1998_LHCUeberblick_CERN_02_ed178e98ea}


  \source{CERN}



\end{frame}


\begin{frame}
  \frametitle{Consumers can also move location}

  Aluminium smelting is an electricity-intensive process. Aluminium
  smelters will often move to locations with cheap and stable
  electricity supplies, such as countries with lots of hydroelectric
  power.  For example, 73\% of Iceland's total power consumption in
  2010 came from aluminium smelting.

  Aluminium costs around US\$ 1500/ tonne to produce. %http://www.lme.com/metals/non-ferrous/aluminium/

  Electricity consumption: 15 MWh/tonne.

  At Germany consumer price of \euro 300 / MWh, this is \euro 4500 / tonne. Uh-oh!!!

  If electricity is 50\% of cost, then need \$750/tonne to go on electricity $\Rightarrow$ 750/15 \$/MWh = 50 \$/MWh.

  %https://arcticecon.wordpress.com/2012/02/15/aluminium-smelting-in-iceland-alcoa-rio-tinto-alcan-century-aluminum-corp/
  %15.7 kWh of electricity to produce 1 kg of aluminium.  Al203 (alumina) -> Al

  % The Hall-Héroult electrolysis process is the major production route for primary aluminium.

  %Due to the nature of the process, power outages have the potential
  %to cause damage to production cells as the molten liquids could
  %solidify in absence of adequate current.  For this reason,
  %production facilities need to be near secure and reliable sources
  %of energy.  step before smelting is bauxite -> alumina. Bauxite is
  %gibbsite Al(OH)3, boehmite γ-AlO(OH), and diaspore α-AlO(OH).


%What is notable is that the aluminium industry’s total power usage
  %amounted to roughly 73% of Iceland’s total power consumption in 2010.

  %https://en.wikipedia.org/wiki/Economy_of_Iceland#Aluminium

  %https://agmetalminer.com/2015/11/24/power-costs-the-production-primary-aluminum/

  %making domestic electricity costs some 44% of the sales price of
  %ingot and may explain why Chinese smelters combined with low
  %capacity utilization due to reduced demand are widely reported

  %http://www.crugroup.com/about-cru/cruinsight/Aluminium_Smelter_Power_Tariffs_Winners_and_Losers

%Power typically represents more than a third of an aluminium
%smelter's operating costs and is the single most important cost
%variant between smelting assets. As a result access to cheap power is
%one of the key sources of competitive advantage in primary aluminium
%smelting.

  % https://en.wikipedia.org/wiki/List_of_aluminium_smelters

  % THere are some in Germany

  \end{frame}


\begin{frame}{Summary: Consumers and Generators}

{\bf Generators:}  A generator has a \alert{cost} or \alert{supply function} $C(g)$ in \euro/h, which
  gives the costs (of fuel, etc.) for a given rate of electricity
  generation $g$~MW. If the market price is $\lambda$ \euro/MWh, the revenue is $\l g$ and the generator should adjust their
generation rate $g$ to maximise their \alert{net generation surplus}, i.e. their profit:
  \begin{equation*}
    \max_g \left[ \l g - C(g) \right]
  \end{equation*}


{\bf Consumers:} Their \alert{utility} or \alert{value function} $U(d)$ in \euro/h is a
  measure of their benefit for a given consumption rate $d$. For a given price $\lambda$ they adjust their consumption rate $d$ such that their \alert{net surplus} is maximised:
    \begin{equation*}
    \max_d \left[U(d) - \l d \right]
  \end{equation*}

  \end{frame}

\section{Supply and Demand at a Single Node}


\begin{frame}{Setting the quantity and price}

  \alert{Total welfare} (consumer and generator surplus) is maximised
  if the total quantity of electricity is set where the aggregated marginal cost and
  marginal utility curves of all generators and consumers meet.

  If the price is also set from this point, then the individual
  optimal actions of each actor will achieve this result in a perfect
  decentralised market.

  \centering
  \includegraphics[width=8cm]{consumer_producer_surplus.pdf}




\end{frame}



\begin{frame}{The result of optimisation}

  This is the result of maximising the \alert{total economic welfare}, the sum of the consumer and the producer surplus for consumers $b$ with consumption $d_b$ and generators $s$ generating with rate $g_s$:
  \begin{align*}
    \max_{\{d_b\}, \{g_s\}}\left[ \sum_b U_b (d_b)  -  \sum_s C_s (g_s) \right]
  \end{align*}
  subject to the supply equalling the demand in the balance constraint:
  \begin{align*}
    \sum_b d_b -  \sum_s g_s  = 0 \hspace{1cm} \leftrightarrow \hspace{1cm} \l
  \end{align*}
  and any other constraints (e.g. limits on generator capacity, etc.).

  Market price $\l$ is the shadow price of the balance constraint,
  i.e. the cost of supply an extra increment 1~MW of demand.

\end{frame}


\begin{frame}{Why decentralised markets work (in theory)}

  We will now show our main result:

      \begin{alertblock}{Welfare-maximisation through decentralised markets}
        The welfare-maximising combination of production and consumption can be achieved by the decentralised profit-maximising decisions of producers and the utility-maximising decisions of consumers, provided that:
        \begin{itemize}
        \item The market price is equal to the constraint marginal value of the overall supply-balance constraint in the welfare maximisation problem
          \item All producers and consumers are price-takers
        \end{itemize}
      \end{alertblock}

\end{frame}




\begin{frame}{KKT and Welfare Maximisation 1/2}

  Apply KKT now to maximisation of total economic welfare:
  \begin{align*}
    \max_{\{d_b\}, \{g_s\}} f(\{d_b\}, \{g_s\}) = \left[ \sum_b U_b (d_b)  -  \sum_s C_s (g_s) \right]
  \end{align*}
  subject to the balance constraint:
  \begin{align*}
    g(\{d_b\}, \{g_s\}) = \sum_b d_b -  \sum_s g_s  = 0 \hspace{1cm} \leftrightarrow \hspace{1cm} \l
  \end{align*}
  and any other constraints (e.g. limits on generator capacity, etc.).

  Our optimisation variables are $\{x\} = \{d_b\} \cup \{g_s\}$.

  We get from stationarity:
  \begin{align*}
    0 & =   \frac{\d f}{\d d_b} - \sum_b \l^* \frac{\d g}{\d d_b} = U_b'(d_b) - \l^* = 0 \\
    0 & =   \frac{\d f}{\d g_s} - \sum_s \l^* \frac{\d g}{\d g_s} = -C_s'(g_s) + \l^* = 0
  \end{align*}


\end{frame}





\begin{frame}{KKT and Welfare Maximisation 2/2}

  So at the optimal point of maximal total economic welfare we get the
  same result as if everyone maximises their own welfare separately:
  \begin{align*}
 U_b'(d_b) =  \l^*  \\
 C_s'(g_s) = \l^*
  \end{align*}

  This is the CENTRAL result of microeconomics.

  If we have further inequality constraints that are binding, then
  these equations will receive additions with $\m_i^* > 0$.


\end{frame}



\begin{frame}[fragile]
  \frametitle{Power Production Real Example}

  At this website you can see the forecast of load, wind, solar and conventional generation right now in Germany (\href{https://energy-charts.de/power_de.htm}{link}):


  \centering
    \includegraphics[width=11cm]{strom-170531-14.png}


\end{frame}

\begin{frame}[fragile]
  \frametitle{Supply-Demand Curve Real Example}

  A real supply-demand curve for Germany-Austria from 2017 (\href{https://www.epexspot.com/en/market-data/dayaheadauction/curve/auction-aggregated-curve/2017-05-31/DE/14/3}{link})

  \centering
    \includegraphics[width=11cm]{DEAT-170531-14}
\end{frame}




\begin{frame}
  \frametitle{Effect of varying demand for fixed generation}

  \centering
  \includegraphics[width=9cm]{demand-supply}

  \source{Tom Brown}

\end{frame}




\begin{frame}
  \frametitle{Example market 1/3}

  \centering

  \includegraphics[width=9cm]{germany_generation-price-2015-01-06-2015-01-12}

\end{frame}

\begin{frame}
  \frametitle{Example market 2/3}

  \centering
  \includegraphics[width=9cm]{germany_generation-price-2015-03-15-2015-03-31}

\end{frame}

\begin{frame}
  \frametitle{Example market 3/3}

  \centering
  \includegraphics[width=9cm]{germany_generation-price-2015-04-07-2015-04-14}

\end{frame}


\begin{frame}
  \frametitle{Effect of varying renewables: fixed demand, no wind}

  \centering
  \includegraphics[width=9cm]{demand-supply-no_wind}

\end{frame}



\begin{frame}
  \frametitle{Effect of varying renewables: fixed demand, 35~GW wind}

  \includegraphics[width=9cm]{demand-supply-with_wind}

\end{frame}



\begin{frame}
  \frametitle{Spot market price development}

  As a result of so much zero-marginal-cost renewable feed-in, spot
  market prices steadily decreased until 2016. Since then prices have
  been rising due to rising gas and CO$_2$ prices.

  \includegraphics[width=11cm]{price_development-2018.png}

  \source{\href{https://www.agora-energiewende.de/fileadmin2/Projekte/2018/Jahresauswertung_2018/Agora-Annual-Review-2018_Energy-Transition-EN.pdf}{Agora Energiewende}}
\end{frame}


\begin{frame}
  \frametitle{Merit Order Effect}

  To summarise:
  \begin{itemize}
  \item Renewables have zero marginal cost
  \item As a result they enter at the bottom of the merit order, reducing the price at which the market clears
  \item This pushes non-CHP gas and hard coal out of the market
  \item This is unfortunate, because among the fossil fuels, gas and
    hard coal are the most flexible and produce the \emph{lowest} \co2
    per MWh
  \item It also massively reduces the profits that nuclear and brown coal make
  \item Will there be enough backup power plants for times with no wind/solar?
  \end{itemize}

  This has led to lots of political tension...

\end{frame}

\end{document}
